\documentclass[aspectratio=169, xcolor=table]{beamer}
\usepackage{enumitem}
%\setlist[enumerate]{label=\arabic*.}% see https://tex.stackexchange.com/questions/24371/does-enumitem-conflict-with-beamer-for-lists/24491
\setenumerate[1]{label=\arabic*.}\newcounter{ResumeEnumerate}% see https://latex.org/forum/viewtopic.php?t=9362&start=10
\setitemize{label=\usebeamerfont*{itemize item}% see https://tex.stackexchange.com/questions/24371/does-enumitem-conflict-with-beamer-for-lists/24491
  \usebeamercolor[fg]{itemize item}
  \usebeamertemplate{itemize item}}
\usepackage[strings]{underscore} % allows hyphenation at underscores
\usepackage{datetime2}\DTMsetdatestyle{iso}
\usepackage{pgffor}

%\usepackage{ifthen}
%\newboolean{movies}\setboolean{movies}{true} % whether to embed movies - MAY NEED TO RUN TWICE!
%\usepackage{colortbl}
%\usepackage[table]{xcolor}    % loads also �colortbl�
%\usepackage{colortbl}
\definecolor{lightblue}{rgb}{0.93,0.95,1.0}
%\rowcolors{2}{blue!4}{white}
%\rowcolors{1}{lightblue}{white}

%\usepackage{transparent}

% \usepackage{beamerthemesplit} // Activate for custom appearance
\usepackage{qrcode}
\usepackage{tabularx}
%\usepackage{media9}
\usepackage{sistyle} % TODO: replace with siunitx?

% - from  CTAN: http://www.ctan.org/tex-archive/macros/latex/contrib/movie15/
% see http://darkwing.uoregon.edu/~noeckel/PDFmovie.html
% [autoplay,autoresume,poster,repeat]
%\usepackage{movie15}
%\newcommand{\run}[1]{\small{\href{run:#1}{$\triangleright$}}}
%\ifthenelse{\boolean{movies}}
%{
%\newcommand{\postermoviewidth}[3]{\includemovie[mimetype=video/mpeg, controls=true, %autoplay, 
%autopause, repeat, text={\includegraphics[width=#3]{#1}}]{#3}{}{#2}\\[-0.7ex]\hfill\run{#2}} % movie should play on page when clicked (this movie is embedded in the PDF). The triangle is a fallback to run the movie from a separate file in another window.
%}
%{
%\newcommand{\postermoviewidth}[3]{\href{run:#2}{\includegraphics[width=#3]{#1}}\\[-0.7ex]\hfill\tiny{\textbf{\href{run:#2}{\phantom{Click here to play the movie in a separate window}
%}}}\hfill \small{\href{run:#2}{$\triangleright$}}
%} % movie plays from a separate file in a separate window (not embedded in PDF).
%}


\usepackage{tikz}

\tikzset{
    state/.style={
           rectangle,
           rounded corners,
           draw=black, ultra thick,
           minimum height=2em,
           inner sep=2pt,
           text centered,
           text width=40ex
           },
}

\usepackage[percent]{overpic}

\urlstyle{sf} % rm, sf, tt or same
% from http://www.kronto.org/thesis/tips/url-formatting.html
%% Define a new 'leo' style for the package that will use a smaller font.
\makeatletter
\def\url@leostyle{%
  \@ifundefined{selectfont}{\def\UrlFont{\sf}}{\def\UrlFont{\normalsize\ttfamily}}}
\makeatother
%% Now actually use the newly defined style.
\urlstyle{leo}
\usepackage[scaled=.9]{inconsolata} % for texttt


\def\colorize<#1>{%
           \temporal<#1>{\color{black!25}}{\color{red!75!black}}{\color{black}}}




%%%%%%%%%%%%%%%%%%%%%%%%%%%%%%%%%%%%%%%%%%%%%%%%%%%%

% commands for making slides with figures

\newcommand{\pwidth}{0.495\textwidth}
\newcommand{\plotter}{}

\newcommand{\monthlymean}{1993-2017_mean_month_}
\newcommand{\years}{\monthlymean}
%\newcommand{\years}{2016-2017}

\newcommand{\plotnocbar}[2][.495\textwidth]{% #1=width (optional), #2=file
\includegraphics[width=#1, trim=45 90 45 7, clip]{#2}
}

\newcommand{\cbar}[2][.495\textwidth]{% #1=width (optional), #2=file
\includegraphics[width=#1, trim=30 7 30 417, clip]{#2}
}

\newcommand{\cbarrot}[2][.495\textwidth]{% #1=width (optional), #2=file
\includegraphics[width=#1, trim=30 7 30 417, clip, angle=90]{#2}
}

% these 3 commands are identical apart from bb
\newcommand{\iceplot}[5][.495\textwidth]{% #1=width (optional), #2=view, #3=variable, #4=experiment, #5=date
\includegraphics[width=#1, trim=45 9 45 7, clip]{figs/#2_#3_#4_\years #5_200dpi.png}
}
\newcommand{\iceplotnocbar}[5][.495\textwidth]{% #1=width (optional), #2=view, #3=variable, #4=experiment, #5=date
%\includegraphics[width=#1, trim=45 90 45 7, clip]{figs/#2_#3_#4_\years #5_200dpi.png}
\plotnocbar[#1]{figs/#2_#3_#4_\years #5_200dpi.png}
}
\newcommand{\icecbar}[5][.495\textwidth]{% #1=width (optional), #2=view, #3=variable, #4=experiment, #5=date
%\includegraphics[width=#1, trim=30 7 30 417, clip]{figs/#2_#3_#4_\years #5_200dpi.png}
\cbar[#1]{figs/#2_#3_#4_\years #5_200dpi.png}
}
\newcommand{\icecbarrot}[5][.495\textwidth]{% #1=width (optional), #2=view, #3=variable, #4=experiment, #5=date
%\includegraphics[width=#1, trim=30 7 30 417, clip, angle=90]{figs/#2_#3_#4_\years #5_200dpi.png}
\cbarrot[#1]{figs/#2_#3_#4_\years #5_200dpi.png}
}


\newcommand{\sicthreepanels}[6]{% #1=view, #2=variable, #3=experiment1, #4=experiment2, #5=experiment3, #6=date
\frame{
\hfill\iceplotnocbar[.32\textwidth]{#1}{#2}{#3}{#6}%
\iceplotnocbar[.32\textwidth]{#1}{#2}{#4}{#6}%
\iceplotnocbar[.32\textwidth]{#1}{#2}{#5}{#6}\hfill\phantom{\quad}\\
\phantom{\quad}\hfill\icecbar{#1}{#2}{#3}{#6}\hfill\phantom{\quad}%
}}

\newcommand{\sicthreepanelsall}[5]{% #1=view, #2=variable, #3=experiment1, #4=experiment2, #5=experiment3
\renewcommand{\years}{\monthlymean}
\sicthreepanels{#1}{#2}{#3}{#4}{#5}{01}
\sicthreepanels{#1}{#2}{#3}{#4}{#5}{02}
\sicthreepanels{#1}{#2}{#3}{#4}{#5}{03}
\sicthreepanels{#1}{#2}{#3}{#4}{#5}{04}
\sicthreepanels{#1}{#2}{#3}{#4}{#5}{05}
\sicthreepanels{#1}{#2}{#3}{#4}{#5}{06}
\sicthreepanels{#1}{#2}{#3}{#4}{#5}{07}
\sicthreepanels{#1}{#2}{#3}{#4}{#5}{08}
\sicthreepanels{#1}{#2}{#3}{#4}{#5}{09}
\sicthreepanels{#1}{#2}{#3}{#4}{#5}{10}
\sicthreepanels{#1}{#2}{#3}{#4}{#5}{11}
\sicthreepanels{#1}{#2}{#3}{#4}{#5}{12}
}


%%\newcommand{\sicfourpanels}[7]{% #1=view, #2=variable, #3=experiment1, #4=experiment2, #5=experiment3, #6=experiment4, #7=date
%\newcommand{\sicfourpanels}[8][.28\textwidth]{% #1=width (optional), #2=view, #3=variable, #4=experiment1, #5=experiment2, #6=experiment3, #7=experiment4, #8=date
%\frame{
%\hfill\iceplotnocbar[#1]{#2}{#3}{#4}{#8}%
%\iceplotnocbar[#1]{#2}{#3}{#5}{#8}\hfill\phantom{\quad}\\
%\hfill\iceplotnocbar[#1]{#2}{#3}{#6}{#8}%
%\iceplotnocbar[#1]{#2}{#3}{#7}{#8}\hfill\phantom{\quad}\\
%\phantom{\quad}\hfill\icecbar{#2}{#3}{#4}{#8}\hfill\phantom{\quad}%
%}}

\newcommand{\sicfourpanels}[9]{% #1=view, #2=variable, #3=experiment1, #4=experiment2, #5=experiment3, #6=experiment4, #7=date, #8=datestr, #9=title
\frame{
\frametitle{#9\\#8\phantom{y}}
\vspace{-10ex}
\begin{minipage}[c]{0.2\textwidth}
\begin{flushright}
\textbf{GIOMAS}\\
{\small (DA model)}\\
\vspace{5ex}
\textbf{0.25$^\circ$}\\
{\small ACCESS-OM2-025}
\vspace{5ex}
\end{flushright}
\end{minipage}%
\hfill
\begin{minipage}[c]{0.58\textwidth}
\iceplotnocbar[.49\textwidth]{#1}{#2}{#3}{#7}\hfill%
\iceplotnocbar[.49\textwidth]{#1}{#2}{#4}{#7}\\
\iceplotnocbar[.49\textwidth]{#1}{#2}{#5}{#7}\hfill%
\iceplotnocbar[.49\textwidth]{#1}{#2}{#6}{#7}\\
\phantom{\quad}\hfill\icecbar[0.8\textwidth]{#1}{#2}{#3}{#7}\hfill\phantom{\quad}%
\end{minipage}%
\hfill
\begin{minipage}[c]{0.2\textwidth}
\begin{flushleft}
\textbf{1$^\circ$}\\
{\small ACCESS-OM2}\\
\vspace{4ex}
\textbf{0.1$^\circ$}\\
{\small ACCESS-OM2-01}
\vspace{7ex}
\end{flushleft}
\end{minipage}
}}


%\sicfourpanels{Amundsen-Bellingshausen}{aice}{025deg_jra55_iaf_amoctopo_cycle1}

%\sicfourpanels{SH}{hi_m_mm}{GIOMAS}{1deg_jra55_iaf_omip2-fixed_cycle1}{025deg_jra55_iaf_amoctopo_cycle1}{01deg_jra55v140_iaf_cycle1}{01}{1993--2017\\[-1ex]January mean}{SI thickness}
%
%
%\end{document}

\newcommand{\sicfourpanelsall}[6]{% #1=view, #2=variable, #3=experiment1, #4=experiment2, #5=experiment3, #6=experiment4
\renewcommand{\years}{\monthlymean}
%\sicfourpanels{#1}{#2}{#3}{#4}{#5}{#6}{01}
%\sicfourpanels{#1}{#2}{#3}{#4}{#5}{#6}{02}
%\sicfourpanels{#1}{#2}{#3}{#4}{#5}{#6}{03}
%\sicfourpanels{#1}{#2}{#3}{#4}{#5}{#6}{04}
%\sicfourpanels{#1}{#2}{#3}{#4}{#5}{#6}{05}
%\sicfourpanels{#1}{#2}{#3}{#4}{#5}{#6}{06}
%\sicfourpanels{#1}{#2}{#3}{#4}{#5}{#6}{07}
%\sicfourpanels{#1}{#2}{#3}{#4}{#5}{#6}{08}
%\sicfourpanels{#1}{#2}{#3}{#4}{#5}{#6}{09}
%\sicfourpanels{#1}{#2}{#3}{#4}{#5}{#6}{10}
%\sicfourpanels{#1}{#2}{#3}{#4}{#5}{#6}{11}
%\sicfourpanels{#1}{#2}{#3}{#4}{#5}{#6}{12}
\sicfourpanels{#1}{#2}{#3}{#4}{#5}{#6}{01}{1993--2017\\[-1ex]January mean}{SI thickness}
\sicfourpanels{#1}{#2}{#3}{#4}{#5}{#6}{02}{1993--2017\\[-1ex]February mean}{SI thickness}
\sicfourpanels{#1}{#2}{#3}{#4}{#5}{#6}{03}{1993--2017\\[-1ex]March mean}{SI thickness}
\sicfourpanels{#1}{#2}{#3}{#4}{#5}{#6}{04}{1993--2017\\[-1ex]April mean}{SI thickness}
\sicfourpanels{#1}{#2}{#3}{#4}{#5}{#6}{05}{1993--2017\\[-1ex]May mean}{SI thickness}
\sicfourpanels{#1}{#2}{#3}{#4}{#5}{#6}{06}{1993--2017\\[-1ex]June mean}{SI thickness}
\sicfourpanels{#1}{#2}{#3}{#4}{#5}{#6}{07}{1993--2017\\[-1ex]July mean}{SI thickness}
\sicfourpanels{#1}{#2}{#3}{#4}{#5}{#6}{08}{1993--2017\\[-1ex]August mean}{SI thickness}
\sicfourpanels{#1}{#2}{#3}{#4}{#5}{#6}{09}{1993--2017\\[-1ex]September mean}{SI thickness}
\sicfourpanels{#1}{#2}{#3}{#4}{#5}{#6}{10}{1993--2017\\[-1ex]October mean}{SI thickness}
\sicfourpanels{#1}{#2}{#3}{#4}{#5}{#6}{11}{1993--2017\\[-1ex]November mean}{SI thickness}
\sicfourpanels{#1}{#2}{#3}{#4}{#5}{#6}{12}{1993--2017\\[-1ex]December mean}{SI thickness}
}

\newcommand{\sicsixpanelsJRAcompare}[9][.32\textwidth]{% #1=width (optional), #2=view, #3=model view, var & experiment, #4-#8=JRAvars, #9=date
\frame{
\hfill%
\includegraphics[width=#1, trim=45 9 45 7, clip]{figs/#3_#9_200dpi.png}\hfill%
\includegraphics[width=#1, trim=45 9 45 7, clip]{figs/#2_#4_d_JRA55-do_#9_200dpi.png}\hfill%
\includegraphics[width=#1, trim=45 9 45 7, clip]{figs/#2_#5_d_JRA55-do_#9_200dpi.png}\hfill\phantom{x}%
\\\hfill%
\includegraphics[width=#1, trim=45 9 45 7, clip]{figs/#2_#6_d_JRA55-do_#9_200dpi.png}\hfill%
\includegraphics[width=#1, trim=45 9 45 7, clip]{figs/#2_#7_d_JRA55-do_#9_200dpi.png}\hfill%
\includegraphics[width=#1, trim=45 9 45 7, clip]{figs/#2_#8_d_JRA55-do_#9_200dpi.png}\hfill\phantom{x}%
}}

\newcommand{\siccompare}[3]{% #1=view, #2=experiment, #3=date
\frame{
\iceplotnocbar{#1}{aice_m_mm}{#2}{#3}\hfill%
\iceplotnocbar{#1}{aice}{obs}{#3}\\
\phantom{\quad}\hfill\icecbar{#1}{aice_m_mm}{#2}{#3}\hfill\phantom{\quad}%
}}

%\newcommand{\siccomparefour}[5]{% #1=view, #2=experiment1, #3=experiment2, #4=experiment3, #5=date
%\frame{
%\hfill\iceplotnocbar[.28\textwidth]{#1}{aice_m_mm}{#2}{#5}%
%\iceplotnocbar[.28\textwidth]{#1}{aice_m_mm}{#3}{#5}\hfill\phantom{\quad}\\
%\hfill\iceplotnocbar[.28\textwidth]{#1}{aice_m_mm}{#4}{#5}%
%\iceplotnocbar[.28\textwidth]{#1}{aice}{obs}{#5}\hfill\phantom{\quad}\\
%\phantom{\quad}\hfill\icecbar{#1}{aice_m_mm}{#2}{#5}\hfill\phantom{\quad}%
%}}

%\newcommand{\siccomparefour}[7]{% #1=view, #2=experiment1, #3=experiment2, #4=experiment3, #5=date, #6=datestr, #7=title
%\frame{
%\frametitle{#7\hfill #6}
%
%\vspace{-4ex}
%\hfill NOAA passive microwave \iceplotnocbar[.28\textwidth]{#1}{aice}{obs}{#5}%
%\iceplotnocbar[.28\textwidth]{#1}{aice_m_mm}{#2}{#5} 1$^\circ$\hfill\phantom{\quad}\\
%\hfill 0.25$^\circ$ \iceplotnocbar[.28\textwidth]{#1}{aice_m_mm}{#3}{#5}%
%\iceplotnocbar[.28\textwidth]{#1}{aice_m_mm}{#4}{#5} 0.1$^\circ$\hfill\phantom{\quad}\\
%\phantom{\quad}\hfill\icecbar{#1}{aice_m_mm}{#2}{#5}\hfill\phantom{\quad}%
%}}

\newcommand{\siccomparefour}[7]{% #1=view, #2=experiment1, #3=experiment2, #4=experiment3, #5=date, #6=datestr, #7=title
\frame{
\frametitle{#7\\#6\phantom{y}}
\vspace{-10ex}
\begin{minipage}[c]{0.2\textwidth}
\begin{flushright}
\textbf{Observational estimate}\\
{\small (NOAA passive microwave)}\\
\vspace{5ex}
\textbf{0.25$^\circ$}\\
{\small ACCESS-OM2-025}
\vspace{5ex}
\end{flushright}
\end{minipage}%
\hfill
\begin{minipage}[c]{0.58\textwidth}
\iceplotnocbar[.49\textwidth]{#1}{aice}{obs}{#5}\hfill%
\iceplotnocbar[.49\textwidth]{#1}{aice_m_mm}{#2}{#5}\\
\iceplotnocbar[.49\textwidth]{#1}{aice_m_mm}{#3}{#5}\hfill%
\iceplotnocbar[.49\textwidth]{#1}{aice_m_mm}{#4}{#5}\\
\phantom{\quad}\hfill\icecbar[0.8\textwidth]{#1}{aice_m_mm}{#2}{#5}\hfill\phantom{\quad}%
\end{minipage}%
\hfill
\begin{minipage}[c]{0.2\textwidth}
\begin{flushleft}
\textbf{1$^\circ$}\\
{\small ACCESS-OM2}\\
\vspace{5ex}
\textbf{0.1$^\circ$}\\
{\small ACCESS-OM2-01}
\vspace{5ex}
\end{flushleft}
\end{minipage}
}}

\newcommand{\siccomparethreecycles}[3]{% #1=view, #2=experiment, #3=date
\frame{
\hfill\iceplotnocbar[.28\textwidth]{#1}{aice}{obs}{#3}%
\iceplotnocbar[.28\textwidth]{#1}{aice_m_mm}{#2_cycle1}{#3}\hfill\phantom{\quad}\\
\hfill\iceplotnocbar[.28\textwidth]{#1}{aice_m_mm}{#2_cycle2}{#3}%
\iceplotnocbar[.28\textwidth]{#1}{aice_m_mm}{#2_cycle3}{#3}\hfill\phantom{\quad}\\
\phantom{\quad}\hfill\icecbar{#1}{aice_m_mm}{#2_cycle1}{#3}\hfill\phantom{\quad}%
}}

\newcommand{\siccomparethreecyclesdiff}[3]{% #1=view, #2=experiment, #3=date
\frame{
\hfill\iceplotnocbar[.28\textwidth]{#1}{aice}{obs}{#3}%
\iceplotnocbar[.28\textwidth]{#1}{aice_m_mm}{#2_cycle1}{#3}\hfill\phantom{\quad}\\
\hfill\iceplotnocbar[.28\textwidth]{#1}{aice_m_mm}{#2_cycle2-cycle1}{#3}%
\iceplotnocbar[.28\textwidth]{#1}{aice_m_mm}{#2_cycle3-cycle1}{#3}\hfill\phantom{\quad}\\
\icecbar{#1}{aice_m_mm}{#2_cycle1}{#3}\hfill\icecbar{#1}{aice_m_mm}{#2_cycle2-cycle1}{#3}
%\hfill\phantom{\quad}%
}}

\newcommand{\siccomparefivecycles}[3]{% #1=view, #2=experiment, #3=date
\frame{
\hfill\iceplotnocbar[.28\textwidth]{#1}{aice}{obs}{#3}
\iceplotnocbar[.28\textwidth]{#1}{aice_m_mm}{#2_cycle1}{#3}
\iceplotnocbar[.28\textwidth]{#1}{aice_m_mm}{#2_cycle2}{#3}\hfill\phantom{\quad}\\
\hfill\iceplotnocbar[.28\textwidth]{#1}{aice_m_mm}{#2_cycle3}{#3}
\iceplotnocbar[.28\textwidth]{#1}{aice_m_mm}{#2_cycle4}{#3}
\iceplotnocbar[.28\textwidth]{#1}{aice_m_mm}{#2_cycle5}{#3}\hfill\phantom{\quad}\\
%\iceplotnocbar[.28\textwidth]{#1}{aice_m_mm}{#2_cycle6}{#3}\hfill\phantom{\quad}\\
\phantom{\quad}\hfill\icecbar{#1}{aice_m_mm}{#2_cycle1}{#3}\hfill\phantom{\quad}%
}}

\newcommand{\siccomparefivecyclesdiff}[3]{% #1=view, #2=experiment, #3=date
\frame{
\hfill\iceplotnocbar[.28\textwidth]{#1}{aice}{obs}{#3}
\iceplotnocbar[.28\textwidth]{#1}{aice_m_mm}{#2_cycle1}{#3}
\iceplotnocbar[.28\textwidth]{#1}{aice_m_mm}{#2_cycle2-cycle1}{#3}\hfill\phantom{\quad}\\
\hfill\iceplotnocbar[.28\textwidth]{#1}{aice_m_mm}{#2_cycle3-cycle1}{#3}
\iceplotnocbar[.28\textwidth]{#1}{aice_m_mm}{#2_cycle4-cycle1}{#3}
\iceplotnocbar[.28\textwidth]{#1}{aice_m_mm}{#2_cycle5-cycle1}{#3}\hfill\phantom{\quad}\\
%\iceplotnocbar[.28\textwidth]{#1}{aice_m_mm}{#2_cycle6-cycle1}{#3}\hfill\phantom{\quad}\\
%\phantom{\quad}\hfill
\icecbar{#1}{aice_m_mm}{#2_cycle1}{#3}\hfill\icecbar{#1}{aice_m_mm}{#2_cycle2-cycle1}{#3}
%\hfill\phantom{\quad}%
}}

\newcommand{\siccomparesixcycles}[3]{% #1=view, #2=experiment, #3=date
\frame{
\hfill\iceplotnocbar[.28\textwidth]{#1}{aice_m_mm}{#2_cycle1}{#3}
\iceplotnocbar[.28\textwidth]{#1}{aice_m_mm}{#2_cycle2}{#3}
\iceplotnocbar[.28\textwidth]{#1}{aice_m_mm}{#2_cycle3}{#3}\hfill\phantom{\quad}\\
\hfill\iceplotnocbar[.28\textwidth]{#1}{aice_m_mm}{#2_cycle4}{#3}
\iceplotnocbar[.28\textwidth]{#1}{aice_m_mm}{#2_cycle5}{#3}
\iceplotnocbar[.28\textwidth]{#1}{aice_m_mm}{#2_cycle6}{#3}\hfill\phantom{\quad}\\
\phantom{\quad}\hfill\icecbar{#1}{aice_m_mm}{#2_cycle1}{#3}\hfill\phantom{\quad}%
}}

\newcommand{\siccomparesixcyclesdiff}[3]{% #1=view, #2=experiment, #3=date
\frame{
\hfill\iceplotnocbar[.28\textwidth]{#1}{aice_m_mm}{#2_cycle1}{#3}
\iceplotnocbar[.28\textwidth]{#1}{aice_m_mm}{#2_cycle2-cycle1}{#3}
\iceplotnocbar[.28\textwidth]{#1}{aice_m_mm}{#2_cycle3-cycle1}{#3}\hfill\phantom{\quad}\\
\hfill\iceplotnocbar[.28\textwidth]{#1}{aice_m_mm}{#2_cycle4-cycle1}{#3}
\iceplotnocbar[.28\textwidth]{#1}{aice_m_mm}{#2_cycle5-cycle1}{#3}
\iceplotnocbar[.28\textwidth]{#1}{aice_m_mm}{#2_cycle6-cycle1}{#3}\hfill\phantom{\quad}\\
%\phantom{\quad}\hfill
\icecbar{#1}{aice_m_mm}{#2_cycle1}{#3}\hfill\icecbar{#1}{aice_m_mm}{#2_cycle2-cycle1}{#3}
%\hfill\phantom{\quad}%
}}

\newcommand{\siccompareall}[3]{% #1=view, #2=experiment, #3=cycle
\renewcommand{\years}{\monthlymean}
\siccompare{#1}{#2_cycle#3}{01}
\siccompare{#1}{#2_cycle#3}{02}
\siccompare{#1}{#2_cycle#3}{03}
\siccompare{#1}{#2_cycle#3}{04}
\siccompare{#1}{#2_cycle#3}{05}
\siccompare{#1}{#2_cycle#3}{06}
\siccompare{#1}{#2_cycle#3}{07}
\siccompare{#1}{#2_cycle#3}{08}
\siccompare{#1}{#2_cycle#3}{09}
\siccompare{#1}{#2_cycle#3}{10}
\siccompare{#1}{#2_cycle#3}{11}
\siccompare{#1}{#2_cycle#3}{12}
}

%
%\newcommand{\siccompareall}[3]{% #1=view, #2=experiment, #3=cycle
%\siccompare{#1}{#2_cycle#3}{01}
%\siccompare{#1}{#2_cycle#3}{02}
%\siccompare{#1}{#2_cycle#3}{03}
%\siccompare{#1}{#2_cycle#3}{04}
%\siccompare{#1}{#2_cycle#3}{05}
%\siccompare{#1}{#2_cycle#3}{06}
%\siccompare{#1}{#2_cycle#3}{07}
%\siccompare{#1}{#2_cycle#3}{08}
%\siccompare{#1}{#2_cycle#3}{09}
%\siccompare{#1}{#2_cycle#3}{10}
%\siccompare{#1}{#2_cycle#3}{11}
%\siccompare{#1}{#2_cycle#3}{12}
%}



\newcommand{\siccomparethreecyclesdiffall}[3]{% #1=view, #2=experiment
\renewcommand{\years}{\monthlymean}
\siccomparethreecyclesdiff{#1}{#2}{01}
\siccomparethreecyclesdiff{#1}{#2}{02}
\siccomparethreecyclesdiff{#1}{#2}{03}
\siccomparethreecyclesdiff{#1}{#2}{04}
\siccomparethreecyclesdiff{#1}{#2}{05}
\siccomparethreecyclesdiff{#1}{#2}{06}
\siccomparethreecyclesdiff{#1}{#2}{07}
\siccomparethreecyclesdiff{#1}{#2}{08}
\siccomparethreecyclesdiff{#1}{#2}{09}
\siccomparethreecyclesdiff{#1}{#2}{10}
\siccomparethreecyclesdiff{#1}{#2}{11}
\siccomparethreecyclesdiff{#1}{#2}{12}
}

\newcommand{\siccomparesixcyclesall}[3]{% #1=view, #2=experiment
\renewcommand{\years}{\monthlymean}
\siccomparesixcycles{#1}{#2}{01}
\siccomparesixcycles{#1}{#2}{02}
\siccomparesixcycles{#1}{#2}{03}
\siccomparesixcycles{#1}{#2}{04}
\siccomparesixcycles{#1}{#2}{05}
\siccomparesixcycles{#1}{#2}{06}
\siccomparesixcycles{#1}{#2}{07}
\siccomparesixcycles{#1}{#2}{08}
\siccomparesixcycles{#1}{#2}{09}
\siccomparesixcycles{#1}{#2}{10}
\siccomparesixcycles{#1}{#2}{11}
\siccomparesixcycles{#1}{#2}{12}
}

\newcommand{\siccomparesixcyclesdiffall}[3]{% #1=view, #2=experiment
\renewcommand{\years}{\monthlymean}
\siccomparesixcyclesdiff{#1}{#2}{01}
\siccomparesixcyclesdiff{#1}{#2}{02}
\siccomparesixcyclesdiff{#1}{#2}{03}
\siccomparesixcyclesdiff{#1}{#2}{04}
\siccomparesixcyclesdiff{#1}{#2}{05}
\siccomparesixcyclesdiff{#1}{#2}{06}
\siccomparesixcyclesdiff{#1}{#2}{07}
\siccomparesixcyclesdiff{#1}{#2}{08}
\siccomparesixcyclesdiff{#1}{#2}{09}
\siccomparesixcyclesdiff{#1}{#2}{10}
\siccomparesixcyclesdiff{#1}{#2}{11}
\siccomparesixcyclesdiff{#1}{#2}{12}
}

\newcommand{\siccomparefivecyclesdiffall}[3]{% #1=view, #2=experiment
\renewcommand{\years}{\monthlymean}
\siccomparefivecyclesdiff{#1}{#2}{01}
\siccomparefivecyclesdiff{#1}{#2}{02}
\siccomparefivecyclesdiff{#1}{#2}{03}
\siccomparefivecyclesdiff{#1}{#2}{04}
\siccomparefivecyclesdiff{#1}{#2}{05}
\siccomparefivecyclesdiff{#1}{#2}{06}
\siccomparefivecyclesdiff{#1}{#2}{07}
\siccomparefivecyclesdiff{#1}{#2}{08}
\siccomparefivecyclesdiff{#1}{#2}{09}
\siccomparefivecyclesdiff{#1}{#2}{10}
\siccomparefivecyclesdiff{#1}{#2}{11}
\siccomparefivecyclesdiff{#1}{#2}{12}
}


% for perturbation experiments %%%%%%%%%%%%%%%%


\newcommand{\sicsixpanelspertcompare}[9][.29\textwidth]{% #1=width (optional), #2=view, #3=model var & control, #4-#8=perturbations, #9=month
\frame{
\hfill%
%SH_aice_m_mm_1deg_jra55_iaf_ensemble_turning_angle_9_-control_1993-2017_mean_month_12_200dpi
\includegraphics[width=#1, trim=45 90 45 7, clip]{{figs/#2_aice_obs_1993-2017_mean_month_#9_200dpi}.png}\hfill% obs
%\includegraphics[width=#1, trim=45 90 45 7, clip]{figs/#2_#3_1993-2017_mean_month_#9_200dpi.png}\hfill% control
\includegraphics[width=#1, trim=45 90 45 7, clip]{{figs/#2_#3#4_1993-2017_mean_month_#9_200dpi}.png}\hfill%
\includegraphics[width=#1, trim=45 90 45 7, clip]{{figs/#2_#3#5_1993-2017_mean_month_#9_200dpi}.png}\hfill\phantom{x}%
\\\hfill%
\includegraphics[width=#1, trim=45 90 45 7, clip]{{figs/#2_#3#6_1993-2017_mean_month_#9_200dpi}.png}\hfill%
\includegraphics[width=#1, trim=45 90 45 7, clip]{{figs/#2_#3#7_1993-2017_mean_month_#9_200dpi}.png}\hfill%
\includegraphics[width=#1, trim=45 90 45 7, clip]{{figs/#2_#3#8_1993-2017_mean_month_#9_200dpi}.png}\hfill\phantom{x}\\
%\phantom{\quad}\hfill
\includegraphics[width=.495\textwidth, trim=30 7 30 417, clip]{{figs/#2_#3#4_1993-2017_mean_month_#9_200dpi}.png}%
\hfill\includegraphics[width=.495\textwidth, trim=30 7 30 417, clip]{{figs/#2_#3#8_1993-2017_mean_month_#9_200dpi}.png}%\phantom{\quad}%
}}

%%%%%%%%%%%%%%%%%%%%%%%%%%%%%%%%%%%%%%%%%%%%%%%%%%%%

\newcommand{\sicobssixpanels}[2]{% #1=view, #2=date
\frame{
\hfill\iceplotnocbar[.28\textwidth]{#1}{aice}{obs}{#2}
\iceplotnocbar[.28\textwidth]{#1}{siconc}{mm}{#2}
\iceplotnocbar[.28\textwidth]{#1}{siconca}{mm}{#2}\hfill\phantom{\quad}\\
\hfill\iceplotnocbar[.28\textwidth]{#1}{aice_m_mm}{1deg_jra55_iaf_ensemble}{#2}
\iceplotnocbar[.28\textwidth]{#1}{aice_m_mm}{025deg_jra55_iaf_ensemble}{#2}
\iceplotnocbar[.28\textwidth]{#1}{aice_m_mm}{01deg_jra55v140_iaf_cycle3}{#2}\hfill\phantom{\quad}\\
\phantom{\quad}\hfill\icecbar{#1}{aice_m_mm}{1deg_jra55_iaf_ensemble}{#2}\hfill\phantom{\quad}%
}}

\newcommand{\sicobssixpanelsbias}[2]{% #1=view, #2=date
%\newcommand{\monthlymean}{1993-2017_mean_month_}
%\renewcommand{\years}{1993-2017_bias_month_}
%\renewcommand{\years}{\monthlymean}
\frame{
\hfill\icecbarrot[.28\textwidth]{#1}{aice_m_mm}{1deg_jra55_iaf_ensemble}{#2}%
\iceplotnocbar[.28\textwidth]{#1}{aice}{obs}{#2}
\iceplotnocbar[.28\textwidth]{#1}{siconc}{mm-obs}{#2}
\iceplotnocbar[.28\textwidth]{#1}{siconca}{mm-obs}{#2}%\hfill\phantom{\quad}\\
\icecbarrot[.28\textwidth]{#1}{siconca}{mm-obs}{#2}\\
\hfill\iceplotnocbar[.28\textwidth]{#1}{aice_m_mm}{1deg_jra55_iaf_omip2_cycle3-obs}{#2}
\iceplotnocbar[.28\textwidth]{#1}{aice_m_mm}{025deg_jra55_iaf_omip2_cycle3-obs}{#2}
\iceplotnocbar[.28\textwidth]{#1}{aice_m_mm}{01deg_jra55v140_iaf_cycle3-obs}{#2}%\hfill\phantom{\quad}\\
%\phantom{\quad}\hfill
\icecbarrot[.28\textwidth]{#1}{aice_m_mm}{1deg_jra55_iaf_omip2_cycle3-obs}{#2}%\hfill\phantom{\quad}%
}}


%\newcommand{\iceplotnocbar}[5][.495\textwidth]{% #1=width (optional), #2=view, #3=variable, #4=experiment, #5=date
%\includegraphics[width=#1, trim=45 90 45 7, clip]{figs/#2_#3_#4_\years #5_200dpi.png}
%}

%\newcommand{\plotnocbar}[2][.495\textwidth]{% #1=width (optional), #2=file
%\includegraphics[width=#1, trim=45 90 45 7, clip]{#2}
%}



%\usebackgroundtemplate{{\centering \includegraphics[width=\paperwidth,keepaspectratio=true]{figures/southern-ocean-faint.png}\par}} 
%\usebackgroundtemplate{{\centering \includegraphics[width=1.2\textwidth,keepaspectratio=true]{figures/01deg_jra55v13_ryf8485_spinup6_ACC_oceanspeed_0026-05-28-background.png}\par}} 
%\usebackgroundtemplate{{\centering \includegraphics[width=1.2\textwidth, decodearray={0.1 1.6 0.1 1.6 0.1 1.6}, trim=10 0 0 25, clip, keepaspectratio=true]{figs/Amundsen-Bellingshausen_mld_01deg_jra55v140_iaf_cycle1_2017-01-16_200dpi.png}\par}}  % decodearray doesn't work in Acrobat
%\usebackgroundtemplate{{\centering \includegraphics[width=1.2\textwidth, trim=10 0 0 25, clip, keepaspectratio=true]{Amundsen-Bellingshausen_mld_01deg_jra55v140_iaf_cycle1_2017-01-16_200dpi-darker.png}\par}}  % decodearray doesn't work in Acrobat

%\title[ACCESS-OM2 ocean-ice model: progress and plans\hspace{0.5\textwidth}\textbf{{\footnotesize\url{cosima.org.au}}}]
\title%[ACCESS-OM2]
{\LARGE\textbf{ACCESS-OM2 sea ice analysis}}
\author{Andrew Kiss\\
\today\ \DTMcurrenttime\\[4ex]
latest PDF: \url{https://www.dropbox.com/sh/co9ge6vnqzyjie1/AABXBvaKQhtSvWHA5tTWSfEpa}\\[2ex]
%latest source version: \url{https://github.com/aekiss/ice_analysis}\\[2ex]
%all figures: \texttt{/g/data/v45/aek156/notebooks/github/aekiss/ice_analysis/figs}
}
%\vspace{4ex}
\date{}
%\date{\textcolor{white}{AMOS Annual Conference, 8--12 Feb 2021}}

\usetheme[width=0pt]{Marburg}
%\usetheme{Rochester}
%\usetheme{Dresden}
\setbeamertemplate{navigation symbols}{} 

%\rowcolors{1}{lightblue}{white}

\begin{document}

\frame{\titlepage}
\usebackgroundtemplate{} 

\begin{frame}
  \frametitle{Outline}
  \tableofcontents
\end{frame}

\section{Background}

\newcommand{\multifig}[1]% #1=expt
{
\includegraphics<1>[width=1.05\textwidth, bb=67 	160 330 384, clip]{figs/Amundsen-Bellingshausen_hi_#1_2017-01-01_200dpi}%
%\includegraphics<2>[width=1.05\textwidth, bb=67 	160 330 384, clip]{figs/Amundsen-Bellingshausen_hi_#1_2017-01-02_200dpi}%
%\includegraphics<3>[width=1.05\textwidth, bb=67 	160 330 384, clip]{figs/Amundsen-Bellingshausen_hi_#1_2017-01-03_200dpi}%
\includegraphics<2>[width=1.05\textwidth, bb=67 	160 330 384, clip]{figs/Amundsen-Bellingshausen_hi_#1_2017-01-04_200dpi}%
%\includegraphics<5>[width=1.05\textwidth, bb=67 	160 330 384, clip]{figs/Amundsen-Bellingshausen_hi_#1_2017-01-05_200dpi}%
%\includegraphics<6>[width=1.05\textwidth, bb=67 	160 330 384, clip]{figs/Amundsen-Bellingshausen_hi_#1_2017-01-06_200dpi}%
\includegraphics<3>[width=1.05\textwidth, bb=67 	160 330 384, clip]{figs/Amundsen-Bellingshausen_hi_#1_2017-01-07_200dpi}%
%\includegraphics<8>[width=1.05\textwidth, bb=67 	160 330 384, clip]{figs/Amundsen-Bellingshausen_hi_#1_2017-01-08_200dpi}%
%\includegraphics<9>[width=1.05\textwidth, bb=67 	160 330 384, clip]{figs/Amundsen-Bellingshausen_hi_#1_2017-01-09_200dpi}%
\includegraphics<4>[width=1.05\textwidth, bb=67 	160 330 384, clip]{figs/Amundsen-Bellingshausen_hi_#1_2017-01-10_200dpi}%
%
%\includegraphics<1>[width=1.05\textwidth, bb=67 	160 330 384, clip]{figs/Amundsen-Bellingshausen_hi_#1_2017-01-01_200dpi}%
%\includegraphics<2>[width=1.05\textwidth, bb=67 	160 330 384, clip]{figs/Amundsen-Bellingshausen_hi_#1_2017-01-02_200dpi}%
%\includegraphics<3>[width=1.05\textwidth, bb=67 	160 330 384, clip]{figs/Amundsen-Bellingshausen_hi_#1_2017-01-03_200dpi}%
%\includegraphics<4>[width=1.05\textwidth, bb=67 	160 330 384, clip]{figs/Amundsen-Bellingshausen_hi_#1_2017-01-04_200dpi}%
%\includegraphics<5>[width=1.05\textwidth, bb=67 	160 330 384, clip]{figs/Amundsen-Bellingshausen_hi_#1_2017-01-05_200dpi}%
%\includegraphics<6>[width=1.05\textwidth, bb=67 	160 330 384, clip]{figs/Amundsen-Bellingshausen_hi_#1_2017-01-06_200dpi}%
%\includegraphics<7>[width=1.05\textwidth, bb=67 	160 330 384, clip]{figs/Amundsen-Bellingshausen_hi_#1_2017-01-07_200dpi}%
%\includegraphics<8>[width=1.05\textwidth, bb=67 	160 330 384, clip]{figs/Amundsen-Bellingshausen_hi_#1_2017-01-08_200dpi}%
%\includegraphics<9>[width=1.05\textwidth, bb=67 	160 330 384, clip]{figs/Amundsen-Bellingshausen_hi_#1_2017-01-09_200dpi}%
%\includegraphics<10>[width=1.05\textwidth, bb=67 	160 330 384, clip]{figs/Amundsen-Bellingshausen_hi_#1_2017-01-10_200dpi}%
%\includegraphics<11>[width=1.05\textwidth, bb=67 	160 330 384, clip]{figs/Amundsen-Bellingshausen_hi_#1_2017-01-11_200dpi}%
%\includegraphics<12>[width=1.05\textwidth, bb=67 	160 330 384, clip]{figs/Amundsen-Bellingshausen_hi_#1_2017-01-12_200dpi}%
%\includegraphics<13>[width=1.05\textwidth, bb=67 	160 330 384, clip]{figs/Amundsen-Bellingshausen_hi_#1_2017-01-13_200dpi}%
%\includegraphics<14>[width=1.05\textwidth, bb=67 	160 330 384, clip]{figs/Amundsen-Bellingshausen_hi_#1_2017-01-14_200dpi}%
%\includegraphics<15>[width=1.05\textwidth, bb=67 	160 330 384, clip]{figs/Amundsen-Bellingshausen_hi_#1_2017-01-15_200dpi}%
%\includegraphics<16>[width=1.05\textwidth, bb=67 	160 330 384, clip]{figs/Amundsen-Bellingshausen_hi_#1_2017-01-16_200dpi}%
%\includegraphics<17>[width=1.05\textwidth, bb=67 	160 330 384, clip]{figs/Amundsen-Bellingshausen_hi_#1_2017-01-17_200dpi}%
%\includegraphics<18>[width=1.05\textwidth, bb=67 	160 330 384, clip]{figs/Amundsen-Bellingshausen_hi_#1_2017-01-18_200dpi}%
%\includegraphics<19>[width=1.05\textwidth, bb=67 	160 330 384, clip]{figs/Amundsen-Bellingshausen_hi_#1_2017-01-19_200dpi}%
%\includegraphics<20>[width=1.05\textwidth, bb=67 	160 330 384, clip]{figs/Amundsen-Bellingshausen_hi_#1_2017-01-20_200dpi}%
%\includegraphics<21>[width=1.05\textwidth, bb=67 	160 330 384, clip]{figs/Amundsen-Bellingshausen_hi_#1_2017-01-21_200dpi}%
%\includegraphics<22>[width=1.05\textwidth, bb=67 	160 330 384, clip]{figs/Amundsen-Bellingshausen_hi_#1_2017-01-22_200dpi}%
}

\frame{
  \frametitle{COSIMA's global ocean-ice model configurations at \textbf{three resolutions}}
\begin{minipage}[t]{0.32\textwidth}
\textbf{ACCESS-OM2}\\[-3.5ex]
{\small 
\begin{itemize}
\item not eddy-resolving
\itemsep-0.25em 
\item \textbf{1$^\circ$} horizontal grid\\[-1ex]
\hspace{-2ex}{\tiny $360\times300$ cells, 24--111\,km}
\item 50 $z^*$ levels\\[-1ex]
\hspace{-2ex}{\tiny $\Delta z=2.3$--220\,m}
\item fast and cheap\\[-1ex]
\hspace{-2ex}{\tiny $\sim 24$min/yr, 0.1\,kCPU\,hr/yr\\[-1.3ex]on 252 PEs, dt=5400\,s}
\item used in ACCESS-CM2
%\item suits many-century\\ {experiments}
\end{itemize}}
\vspace{-1ex}%
\multifig{1deg_jra55_iaf_omip2-fixed_cycle1}
\end{minipage}
\hfill
\begin{minipage}[t]{0.32\textwidth}
 \textbf{ACCESS-OM2-025}\\[-3.5ex]
{\small
\begin{itemize}
\itemsep-0.25em 
\item eddy ``permitting''
\item \textbf{0.25$^\circ$} horizontal grid\\[-1ex]
\hspace{-2ex}{\tiny $1440\times1080$ cells, 6.0--27.8\,km}
\item 50 $z^*$ levels\\[-1ex]
\hspace{-2ex}{\tiny $\Delta z=2.3$--220\,m}
\item fairly fast, less cheap\\[-1ex]
\hspace{-2ex}{\tiny 105\,min/yr, 4.5\,kCPU\,hr/yr\\[-1.3ex]on 1824 PEs, dt=1800\,s}
%\item suits few-century\\ {experiments}
\item for future ACCESS CM
\end{itemize}}
\vspace{-1ex}%
\multifig{025deg_jra55_iaf_amoctopo_cycle1}
\end{minipage}
\hfill
\begin{minipage}[t]{0.32\textwidth}
\textbf{ACCESS-OM2-01}\\[-3.5ex]
{\small 
\begin{itemize}
\itemsep-0.25em 
\item eddy-rich
\item \textbf{0.1$^\circ$} horizontal grid\\[-1ex]
\hspace{-2ex}{\tiny $3600\times2700$ cells, 2.2--11.1\,km}
\item 75  $z^*$ levels\\[-1ex]
\hspace{-2ex}{\tiny $\Delta z=1.1$--198\,m}
\item slow, expensive\\[-1ex]
\hspace{-2ex}{\tiny 9\,hr/yr, 55--65\,kCPU\,hr/yr\\[-1.3ex]on 5096 PEs, dt=600\,s}
%\item suits multi-decade\\ {experiments}
\item for~Bluelink
\end{itemize}}
\vspace{-1ex}%
\multifig{01deg_jra55v140_iaf_cycle1}
\end{minipage}%\\[2ex]
}

\frame{
\frametitle{ACCESS-OM2 suite model components}
\begin{minipage}[t]{0.4\textwidth}
\vfill
\includegraphics[width=\textwidth]{models-diagram/models-diagram-clip.pdf}
\vfill
\end{minipage}%
\begin{minipage}[t]{0.6\textwidth}
\textbf{Prescribed atmospheric forcing: JRA55-do}
\begin{itemize}
\item JRA55-do~v1.4.0 {\scriptsize (55\,km, 3-hourly)}
\item Derived from JRA55 reanalysis but with adjustments for driving ocean models
\item This is the only component tied to observations (no DA in ice or ocean)
\item JRA55 assimilated thresholded COBE-SST SIC, set to 1 for SIC$>0.55$, and 0 otherwise
\item No feedback to atmosphere $\Rightarrow$ % (e.g.\ infinite atmospheric heat capacity)
 atmospheric control of ice and ocean is stronger than in a coupled atmosphere-ice-ocean model %(e.g.\ ACCESS-CM2)
\end{itemize}
\textbf{Model improvements since v1 {\small (\href{https://doi.org/10.5194/gmd-13-401-2020}{Kiss et al.\ 2020})}}
\begin{itemize}
\item  {\small relative winds for ice stress calculation}\\[-4ex]
\item  {\small latitudinally-varying ocean albedo }\\[-4ex]
\item {\small new JRA55-do version, extended to Jan 2019}\\[-4ex]
\item  {\small improved topography at all resolutions}\\[-4ex]
\end{itemize}
\end{minipage}
}

\section{Control experiments at 1$^\circ$, 0.25$^\circ$ and 0.1$^\circ$}

\frame{
\frametitle{Control experiments at 1$^\circ$, 0.25$^\circ$ and 0.1$^\circ$}
\begin{itemize}
\item Each resolution was run for \textbf{multiple 61-year cycles}
\begin{itemize}
\item Each cycle had 1 Jan 1958 -- 31 Dec 2018 JRA55-do v1.4.0 forcing
\item Cycles differ only in their initial condition
\begin{itemize}
\item[\textbullet] {\small cycle 1 starts from observed climatology (World Ocean Atlas 2013 v2)}
\item[\textbullet] {\small subsequent cycles start from final ocean \& ice state of previous cycle}\\[2ex]
\end{itemize}
\end{itemize}
\item \textbf{ACCESS-OM2:} OMIP-2 experiment, \textbf{6 cycles} at 1$^\circ$ (Hakase Hayashida)%  \texttt{/scratch/v45/aek156/access-om2/archive/1deg_jra55_iaf_omip2-fixed_cycle*}
\item \textbf{ACCESS-OM2-025:} OMIP-2 experiment, \textbf{6 cycles} at 0.25$^\circ$ (Ryan Holmes) % \texttt{/scratch/e14/rmh561/access-om2/archive/025deg_jra55_iaf_amoctopo_cycle*}
\item \textbf{ACCESS-OM2-01:} \textbf{3 cycles} at 0.1$^\circ$\\[2ex] %\texttt{/g/data/cj50/access-om2/raw-output/access-om2-01/01deg_jra55v140_iaf}, \texttt{/g/data/cj50/access-om2/raw-output/access-om2-01/01deg_jra55v140_iaf_cycle2}, \texttt{/scratch/x77/aek156/access-om2/archive/01deg_jra55v140_iaf_cycle3} and \texttt{/scratch/v45/aek156/access-om2/archive/01deg_jra55v140_iaf_cycle3}
%\item figure script: \url{https://github.com/aekiss/ice_analysis}
%\item figure files: \texttt{/g/data/v45/aek156/notebooks/github/aekiss/ice_analysis/figs}
%\item All data will be made public (currently available by request)
\item Runs used new configurations that are more consistent than in v1 {\small (\href{https://doi.org/10.5194/gmd-13-401-2020}{Kiss et al.\ 2020})}\\[2ex]
\end{itemize}
}

\subsection{Timeseries}

\frame{
\frametitle{Timeseries of sea ice extent (area with $>15\%$ concentration), cycle \only<1>{1}\only<2>{2}\only<3>{3}\only<4>{4}\only<5>{5}\only<6>{6}}
\includegraphics<1>[height=0.5\textheight,width=\textwidth]{figs/ice_extent_min_mean_max_all_cycle1.pdf}%
\includegraphics<2>[height=0.5\textheight,width=.978\textwidth]{figs/ice_extent_min_mean_max_all_cycle2.pdf}%
\includegraphics<3>[height=0.5\textheight,width=.978\textwidth]{figs/ice_extent_min_mean_max_all_cycle3.pdf}%
\includegraphics<4>[height=0.5\textheight,width=.978\textwidth]{figs/ice_extent_min_mean_max_all_cycle4.pdf}%
\includegraphics<5>[height=0.5\textheight,width=.978\textwidth]{figs/ice_extent_min_mean_max_all_cycle5.pdf}%
\includegraphics<6>[height=0.5\textheight,width=.978\textwidth]{figs/ice_extent_min_mean_max_all_cycle6.pdf}%
\begin{itemize}
\item 12-month running minimum, mean and maximum extent, compared to observational estimate (\href{https://nsidc.org/data/g02135}{NSIDC Sea Ice Index, v3; Fetterer et al.})
\item Very close tracking of interannual variation in obs, due to data assimilation into JRA55-do reanalysis: \textbf{2016 Antarctic sea ice minimum is captured}
\item Little dependence on cycle number, apart from initial few decades at 1$^\circ$ and 0.25$^\circ$
\item Quantitative comparison is hampered by differing land masks
\end{itemize}
}

\frame{
\frametitle{ACCESS-OM2 outperforms most OMIP-2 models in March ice extent}
ACCESS-OM2, all cycles \& res
\hfill
OMIP-2, mostly 1$^\circ$ (\href{https://doi.org/10.5194/gmd-13-3643-2020}{Tsujino et al., 2020})
\hfill\phantom{x}
\begin{minipage}[c]{0.35\textwidth}
\includegraphics[width=\textwidth]{figs/SH_extent_1958-2018_month_03.pdf}\\[-1.1ex]
\includegraphics[width=\textwidth]{figs/SH_extent_1958-2018_month_09.pdf}\\
\end{minipage}%
\begin{minipage}[c]{0.65\textwidth}
\includegraphics[width=\textwidth]{TsujinoETAL2020a-fig22ghjk}\\
\end{minipage}
%\vspace{-6ex}
{\small
JRA55-do v1.4.0 forcing
\hfill
CORE-II forcing\hfill JRA55-do v1.4.0 forcing\hfill\phantom{x}}
}


\frame{
\frametitle{1993--2017 mean annual cycle of sea ice area and extent}
\includegraphics[width=\textwidth, trim=0 0 0 180, clip]{figs/ice_area_seasonal_clim_cycle3.pdf}\\
\includegraphics[width=\textwidth, trim=0 0 0 180, clip]{figs/ice_extent_seasonal_clim_cycle3.pdf}\\
\hfill {\scriptsize Obs: NOAA/NSIDC G02135 Sea Ice Index v3 (Fetterer et al., 2017)}\\[2ex]
\begin{itemize}
\item modelled ice area and extent are too low at minimum (Feb)
\item very close to obs in March
\item model rises too quickly in Autumn, and extent falls too slowly in Spring
\end{itemize}
}

\subsection{Monthly climatological maps}

\frame{
\frametitle{1993--2017 monthly mean SIC climatology and standard deviation}
\begin{itemize}
\item Top row, L-R: monthly mean SIC in NSIDC passive microwave\\ and 1$^\circ$, 0.25$^\circ$ and 0.1$^\circ$ models
\item Bottom row, L-R: standard deviation in monthly mean SIC in NSIDC\\ and 1$^\circ$, 0.25$^\circ$ and 0.1$^\circ$ models
\item SIC biases are very similar across model resolutions
\item interannual variability of SIC is greatest in marginal ice zone (MIZ; i.e. roughly SIC between 0.15 and 0.80)
\item SIC standard deviation is mostly $<0.4$ so climatological biases are a major component of bias in any particular month
\end{itemize}
}
\renewcommand{\pwidth}{0.23\textwidth}
\renewcommand{\plotter}[2]{% #1=view, #2=date
\frame{
\plotnocbar[\pwidth]{figs/#1_aice_obs_1993-2017_mean_month_#2_200dpi.png}%
\plotnocbar[\pwidth]{figs/#1_aice_m_mm_1deg_jra55_iaf_ensemble_1993-2017_mean_month_#2_200dpi.png}%
\plotnocbar[\pwidth]{figs/#1_aice_m_mm_025deg_jra55_iaf_ensemble_1993-2017_mean_month_#2_200dpi.png}%
\plotnocbar[\pwidth]{figs/#1_aice_m_mm_01deg_jra55v140_iaf_cycle3_1993-2017_mean_month_#2_200dpi.png}%
\cbarrot[\pwidth]{figs/#1_aice_obs_1993-2017_mean_month_#2_200dpi.png}\\
\plotnocbar[\pwidth]{figs/#1_aice_obs_1993-2017_std_month_#2_200dpi.png}%
\plotnocbar[\pwidth]{figs/#1_aice_m_1deg_jra55_iaf_omip2_cycle3_1993-2017_std_month_#2_200dpi.png}%
\plotnocbar[\pwidth]{figs/#1_aice_m_025deg_jra55_iaf_omip2_cycle3_1993-2017_std_month_#2_200dpi.png}%
\plotnocbar[\pwidth]{figs/#1_aice_m_01deg_jra55v140_iaf_cycle3_1993-2017_std_month_#2_200dpi.png}%
\cbarrot[\pwidth]{figs/#1_aice_obs_1993-2017_std_month_#2_200dpi.png}\\
}}
\foreach \m in {01,02,03,04,05,06,07,08,09,10,11,12}{\plotter{SH}{\m}}
%\foreach \m in {01}{\plotter{SH}{\m}}

\frame{
\frametitle{Biases in monthly SIC climatology relative to NSIDC}
\begin{itemize}
\item Top row:\\
-- Left: observational SIC product from NSIDC\\
-- Middle: bias of siconc (COBE-SST SIC) relative to NSIDC\\
-- Right: bias of siconca (COBE-SST SIC thresholded daily at 55\% which was assimilated into the JRA55 reanalysis) relative to NSIDC
\item Bottom row, L-R: model biases at 1$^\circ$, 0.25$^\circ$ and 0.1$^\circ$ relative to NSIDC (all from cycle 3) \textbf{NB: different colour scale!}
\item Model SIC biases are nearly resolution-independent; biases slightly smaller at $0.1^\circ$
\item Strong low bias in summer throughout all of NSIDC extent apart; weaker high bias in outer Ross \& Amundsen Seas \& Maud
\item Patchy winter bias in outer MIZ
\item Model SIC biases are much larger than COBE-SST SIC biases, with unrelated patterns
\end{itemize}
}
%\foreach \m in {01}{\sicobssixpanelsbias{SH}{\m}}
\foreach \m in {01,02,03,04,05,06,07,08,09,10,11,12}{\sicobssixpanelsbias{SH}{\m}}

\frame{
\frametitle{1$^\circ$ biases in monthly SIC climatology relative to NSIDC, siconc, siconca}
\begin{itemize}
\item Top row:\\
-- Left: observational SIC product from NSIDC\\
-- Middle: siconc (COBE-SST SIC)\\
-- Right: \textbf{siconca (COBE-SST SIC thresholded daily at 55\% which was assimilated into the JRA55 reanalysis)}
\item Bottom row, L-R: model bias at 1$^\circ$ (cycle 3) relative to NSIDC, siconc, siconca
\item Model bias is much larger than the difference between obs products
\item \textbf{$\Rightarrow$ Model bias is not due to bias in the COBE-SST SIC (siconca) used in JRA55}
\item ... so we'll plot biases relative to NSIDC from here on
\end{itemize}
}
\renewcommand{\pwidth}{0.28\textwidth}
\renewcommand{\plotter}[2]{% #1=view, #2=date
\frame{
\plotnocbar[\pwidth]{figs/#1_aice_obs_1993-2017_mean_month_#2_200dpi.png}%
\plotnocbar[\pwidth]{figs/#1_siconc_mm_1993-2017_mean_month_#2_200dpi.png}%
\plotnocbar[\pwidth]{figs/#1_siconca_mm_1993-2017_mean_month_#2_200dpi.png}%
\cbarrot[\pwidth]{figs/#1_aice_obs_1993-2017_mean_month_#2_200dpi.png}\\%
%SH_aice_m_mm_1deg_jra55_iaf_omip2_cycle6-siconca_1993-2017_mean_month_07_200dpi.png
%\plotnocbar[\pwidth]{figs/#1_aice_obs_1993-2017_mean_month_#2_200dpi.png}%
%\plotnocbar[\pwidth]{figs/#1_siconc_mm_1993-2017_mean_month_#2_200dpi.png}%
\plotnocbar[\pwidth]{figs/#1_aice_m_mm_1deg_jra55_iaf_omip2_cycle3-obs_1993-2017_mean_month_#2_200dpi.png}%
\plotnocbar[\pwidth]{figs/#1_aice_m_mm_1deg_jra55_iaf_omip2_cycle3-siconc_1993-2017_mean_month_#2_200dpi.png}%
\plotnocbar[\pwidth]{figs/#1_aice_m_mm_1deg_jra55_iaf_omip2_cycle3-siconca_1993-2017_mean_month_#2_200dpi.png}\\%
\cbar[\pwidth]{figs/#1_aice_m_mm_1deg_jra55_iaf_omip2_cycle3-obs_1993-2017_mean_month_#2_200dpi.png}%
\cbar[\pwidth]{figs/#1_aice_m_mm_1deg_jra55_iaf_omip2_cycle3-siconc_1993-2017_mean_month_#2_200dpi.png}%
\cbar[\pwidth]{figs/#1_aice_m_mm_1deg_jra55_iaf_omip2_cycle3-siconca_1993-2017_mean_month_#2_200dpi.png}%
%\cbarrot[\pwidth]{figs/#1_aice_m_mm_1deg_jra55_iaf_omip2_cycle3-siconca_1993-2017_mean_month_#2_200dpi.png}%
%\cbarrot[\pwidth]{figs/#1_aice_m_mm_1deg_jra55_iaf_omip2_cycle3-obs_1993-2017_mean_month_#2_200dpi.png}%
%\plotnocbar[\pwidth]{figs/#1_daidtt_m_mm_1deg_jra55_iaf_omip2_cycle3_1993-2017_thermo_tendency_month_#2_200dpi.png}\\
%\cbar[\pwidth]{figs/#1_aice_obs_1993-2017_mean_tendency_month_#2_200dpi.png}%
%\cbar[\pwidth]{figs/#1_aice_m_mm_1deg_jra55_iaf_omip2_cycle3_1993-2017_mean_tendency_month_#2_200dpi.png}%
%\cbar[\pwidth]{figs/#1_daidtt_m_mm_1deg_jra55_iaf_omip2_cycle3_1993-2017_thermo_tendency_month_#2_200dpi.png}\\
%\plotnocbar[\pwidth]{figs/#1_aice_obs_1993-2017_mean_month_#2_200dpi.png}%
%\plotnocbar[\pwidth]{figs/#1_aice_m_mm_1deg_jra55_iaf_omip2_cycle3_1993-2017_tendency_bias_month_#2_200dpi.png}%
%\plotnocbar[\pwidth]{figs/#1_daidtd_m_mm_1deg_jra55_iaf_omip2_cycle3_1993-2017_dynamics_tendency_month_#2_200dpi.png}\\
%\cbar[\pwidth]{figs/#1_aice_obs_1993-2017_mean_month_#2_200dpi.png}%
%\cbar[\pwidth]{figs/#1_aice_m_mm_025deg_jra55_iaf_omip2_cycle3_1993-2017_tendency_bias_month_#2_200dpi.png}%
%\cbar[\pwidth]{figs/#1_daidtd_m_mm_1deg_jra55_iaf_omip2_cycle3_1993-2017_dynamics_tendency_month_#2_200dpi.png}
}}
%\newcommand{\months}{01,02,03,04,05,06,07,08,09,10,11,12}
%\sicobsbias{SH}{01}
\foreach \m in {01,02,03,04,05,06,07,08,09,10,11,12}{\plotter{SH}{\m}}
%\foreach \m in {01,03,05,07,09,11}{\plotter{SH}{\m}}
%\foreach \m in {01}{\plotter{SH}{\m}}


\subsection{SIC histograms}

\frame{
\frametitle{Histograms: model SIC vs observed SIC}
\begin{itemize}
\item Monthly mean at each model grid cell compared to obs, for each year over 1993--2017 --- assesses whether ice has the right concentration at each place and time (rather than in climatological mean)
\item Model predicts pattern of high SIC much better in winter than summer
\item Marginal ice locations not well predicted (without DA)
\item Higher resolution gives better predictions
\end{itemize}
}


\renewcommand{\pwidth}{0.29\textwidth}
\renewcommand{\plotter}[2]{% #1=view, #2=date
\frame{
\includegraphics[width=\pwidth]{figs/#1_aice_m_1deg_jra55_iaf_omip2_cycle3_hist_vs_obs_1993-2017_month_#2_200dpi}\hfill
\includegraphics[width=\pwidth]{figs/#1_aice_m_025deg_jra55_iaf_omip2_cycle3_hist_vs_obs_1993-2017_month_#2_200dpi}\hfill
\includegraphics[width=\pwidth]{figs/#1_aice_m_01deg_jra55v140_iaf_cycle3_hist_vs_obs_1993-2017_month_#2_200dpi}\\
\iceplotnocbar[\pwidth]{#1}{aice_m_mm}{1deg_jra55_iaf_omip2_cycle3-obs}{#2}\hfill
\iceplotnocbar[\pwidth]{#1}{aice_m_mm}{025deg_jra55_iaf_omip2_cycle3-obs}{#2}\hfill
\iceplotnocbar[\pwidth]{#1}{aice_m_mm}{01deg_jra55v140_iaf_cycle3-obs}{#2}%\hfill\phantom{\quad}\\
%\phantom{\quad}\hfill
\icecbarrot[\pwidth]{#1}{aice_m_mm}{1deg_jra55_iaf_omip2_cycle3-obs}{#2}%\hfill\phantom{\quad}%
%\plotnocbar[\pwidth]{figs/#1_aice_m_mm_1deg_jra55_iaf_omip2_cycle3-obs_1993-2017_mean_month_#2_200dpi.png}%
}}
\foreach \m in {01,02,03,04,05,06,07,08,09,10,11,12}{\plotter{SH}{\m}}
%\foreach \m in {01}{\plotter{SH}{\m}}

%%\renewcommand{\plotter}[1]{% #1=date
%\frame{
%\begin{minipage}[c]{0.72\textwidth}
%\includegraphics[width=0.49\textwidth]{figs/SH_aice_m_1deg_jra55_iaf_omip2_cycle3_hist_vs_obs_1993-2017_month_02_200dpi}
%\includegraphics[width=0.49\textwidth]{figs/SH_aice_m_01deg_jra55v140_iaf_cycle3_hist_vs_obs_1993-2017_month_02_200dpi}\\
%\includegraphics[width=0.49\textwidth]{figs/SH_aice_m_1deg_jra55_iaf_omip2_cycle3_hist_vs_obs_1993-2017_month_09_200dpi}
%\includegraphics[width=0.49\textwidth]{figs/SH_aice_m_01deg_jra55v140_iaf_cycle3_hist_vs_obs_1993-2017_month_09_200dpi}
%\end{minipage}
%\begin{minipage}[c]{0.26\textwidth}
%\textbf{Bivariate~histograms}\\
%monthly mean model SIC vs obs SIC at each grid cell for 1993--2017.\\[2ex]
%Model predicts pattern of high SIC much better in winter than summer.\\[2ex]
%Marginal ice locations not well predicted (without DA).
%%Model performance is much better at high concentration and in winter. 
%\end{minipage}
%}%}
%%\foreach \m in {02,09}{\plotter{\m}}

%\end{document}

\subsection{SIC tendency maps}


\frame{
\frametitle{Monthly climatology of SIC tendency}
\begin{itemize}
\item Top row, L-R: SIC tendency (change in monthly mean SIC between this month and previous), for NSIDC passive microwave and 1$^\circ$, 0.25$^\circ$ and 0.1$^\circ$ models
\item Bottom row:\\
-- Left: NSIDC monthly mean SIC\\
-- the rest: bias (model-obs) in model SIC tendency
\item SIC tendency biases are very similar across model resolutions
\item winter biases associated with differing extent of outer MIZ
\item spring loss too slow in outer ice, too rapid in inner
\item early autumn recovery too strong, late autumn too weak
\end{itemize}
\includegraphics[width=.5\textwidth, trim=0 0 0 180, clip]{figs/ice_area_seasonal_clim_cycle3.pdf}\\
}
\renewcommand{\pwidth}{0.235\textwidth}
\renewcommand{\plotter}[2]{% #1=view, #2=date
\frame{
\plotnocbar[\pwidth]{figs/#1_aice_obs_1993-2017_mean_tendency_month_#2_200dpi.png}%
\plotnocbar[\pwidth]{figs/#1_aice_m_mm_1deg_jra55_iaf_omip2_cycle3_1993-2017_mean_tendency_month_#2_200dpi.png}%
\plotnocbar[\pwidth]{figs/#1_aice_m_mm_025deg_jra55_iaf_omip2_cycle3_1993-2017_mean_tendency_month_#2_200dpi.png}%
\plotnocbar[\pwidth]{figs/#1_aice_m_mm_01deg_jra55v140_iaf_cycle3_1993-2017_mean_tendency_month_#2_200dpi.png}%
\cbarrot[\pwidth]{figs/#1_aice_m_mm_025deg_jra55_iaf_omip2_cycle3_1993-2017_mean_tendency_month_#2_200dpi.png}\\%
\plotnocbar[\pwidth]{figs/#1_aice_obs_1993-2017_mean_month_#2_200dpi.png}%
\plotnocbar[\pwidth]{figs/#1_aice_m_mm_1deg_jra55_iaf_omip2_cycle3_1993-2017_tendency_bias_month_#2_200dpi.png}%
\plotnocbar[\pwidth]{figs/#1_aice_m_mm_025deg_jra55_iaf_omip2_cycle3_1993-2017_tendency_bias_month_#2_200dpi.png}%
\plotnocbar[\pwidth]{figs/#1_aice_m_mm_01deg_jra55v140_iaf_cycle3_1993-2017_tendency_bias_month_#2_200dpi.png}%
\cbarrot[\pwidth]{figs/#1_aice_m_mm_025deg_jra55_iaf_omip2_cycle3_1993-2017_tendency_bias_month_#2_200dpi.png}\\
\cbar[\pwidth]{figs/#1_aice_obs_1993-2017_mean_month_#2_200dpi.png}%\hfill%
}}
%\foreach \m in {01}{\plotter{SH}{\m}}
\foreach \m in {01,02,03,04,05,06,07,08,09,10,11,12}{\plotter{SH}{\m}}

\frame{
\frametitle{Dynamic and thermodynamic contributions to SIC tendency}
\begin{itemize}
\item Top row:\\
-- Left: NSIDC SIC tendency (change in monthly mean SIC between this month and previous)\\
-- Middle: 1$^\circ$ tendency  (change in monthly mean SIC between this month and previous)\\
-- Right: 1$^\circ$ tendency due to thermodynamics (SIC change over the month)
\item Bottom row:\\
-- Left: NSIDC monthly mean SIC\\
-- Middle: 1$^\circ$ tendency bias (model - obs)\\
-- Right: 1$^\circ$ tendency due to dynamics (SIC change over the month)
\item winter tendency is balance between competing thermo \& dynamics
\item thermo dominates spring decrease
\item early autumn recovery driven by thermo, subsequently reduced by dynamics
\end{itemize}
}
\renewcommand{\pwidth}{0.28\textwidth}
\renewcommand{\plotter}[2]{% #1=view, #2=date
\frame{
\plotnocbar[\pwidth]{figs/#1_aice_obs_1993-2017_mean_tendency_month_#2_200dpi.png}%
\plotnocbar[\pwidth]{figs/#1_aice_m_mm_1deg_jra55_iaf_omip2_cycle3_1993-2017_mean_tendency_month_#2_200dpi.png}%
\plotnocbar[\pwidth]{figs/#1_daidtt_m_mm_1deg_jra55_iaf_omip2_cycle3_1993-2017_thermo_tendency_month_#2_200dpi.png}\\
\cbar[\pwidth]{figs/#1_aice_obs_1993-2017_mean_tendency_month_#2_200dpi.png}%
\cbar[\pwidth]{figs/#1_aice_m_mm_1deg_jra55_iaf_omip2_cycle3_1993-2017_mean_tendency_month_#2_200dpi.png}%
\cbar[\pwidth]{figs/#1_daidtt_m_mm_1deg_jra55_iaf_omip2_cycle3_1993-2017_thermo_tendency_month_#2_200dpi.png}\\
\plotnocbar[\pwidth]{figs/#1_aice_obs_1993-2017_mean_month_#2_200dpi.png}%
\plotnocbar[\pwidth]{figs/#1_aice_m_mm_1deg_jra55_iaf_omip2_cycle3_1993-2017_tendency_bias_month_#2_200dpi.png}%
\plotnocbar[\pwidth]{figs/#1_daidtd_m_mm_1deg_jra55_iaf_omip2_cycle3_1993-2017_dynamics_tendency_month_#2_200dpi.png}\\
\cbar[\pwidth]{figs/#1_aice_obs_1993-2017_mean_month_#2_200dpi.png}%
\cbar[\pwidth]{figs/#1_aice_m_mm_025deg_jra55_iaf_omip2_cycle3_1993-2017_tendency_bias_month_#2_200dpi.png}%
\cbar[\pwidth]{figs/#1_daidtd_m_mm_1deg_jra55_iaf_omip2_cycle3_1993-2017_dynamics_tendency_month_#2_200dpi.png}
}}
\foreach \m in {01,02,03,04,05,06,07,08,09,10,11,12}{\plotter{SH}{\m}}
%\foreach \m in {01}{\plotter{SH}{\m}}


%\end{document}

%\frame{
%\frametitle{Monthly climatology of sea ice concentration}
%\begin{itemize}
%\item Top row, L-R: observational SIC products from NSIDC, COBE-SST, and \textbf{COBE-SST thresholded daily at 55\% as input to JRA55 reanalysis}
%\item Bottom row, L-R: model SIC at 1$^\circ$, 0.25$^\circ$ and 0.1$^\circ$ (all from cycle 3)
%\item 
%\end{itemize}
%}
%%\foreach \m in {01,02,03,04,05,06,07,08,09,10,11,12}{\sicobssixpanels{SH}{\m}}
%\foreach \m in {01}{\sicobssixpanels{SH}{\m}}


\subsection{Mass tendency terms}

\frame{
\frametitle{Sea ice mass tendency terms}
\includegraphics[height=.6\textheight]{figs/SH_1deg_jra55_iaf_ensemble_cycle1_1993-2017_mass_terms_seasonal_clim.pdf}\hfill
\includegraphics[height=.55\textheight]{LiHuangLiLiuFan2021a-fig8.jpg}\\
\hfill CMIP6 1979--2014 multi-model mean\\
\hfill (Li et al., 2021)
}

\subsection{Daily maps}

\frame{\frametitle{Example: daily sea ice concentration in Amundsen-Bellingshausen Sea}
\begin{itemize}
\item Very similar daily SIC at the 3 resolutions
\end{itemize}
}
%\newcommand{\sicfourpanelsAB}[8][.28\textwidth]{% #1=width (optional), #2=view, #3=variable, #4=experiment1, #5=experiment2, #6=experiment3, #7=experiment4, #8=date
%\frame{
%\hfill\iceplotnocbar[#1]{#2}{#3}{#4}{#8}%
%\iceplotnocbar[#1]{#2}{#3}{#5}{#8}\hfill\phantom{\quad}\\
%\hfill\iceplotnocbar[#1]{#2}{#3}{#6}{#8}%
%\iceplotnocbar[#1]{#2}{#3}{#7}{#8}\hfill\phantom{\quad}\\
%\phantom{\quad}\hfill\icecbar{#2}{#3}{#4}{#8}\hfill\phantom{\quad}%
%}}

\newcommand{\sicfourpanelsAB}[8]{% #1=view, #2=variable, #3=experiment1, #4=experiment2, #5=experiment3, #6=experiment4, #7=date, #8=title
\frame{
\frametitle{#8\\{\small #7}}
\vspace{-7ex}
\begin{minipage}[c]{0.09\textwidth}
\begin{flushright}
\textbf{Obs}\\
%\textbf{Observational estimate}\\
%{\small (NOAA passive microwave)}\\
\vspace{5ex}
\textbf{0.25$^\circ$}\\
%{\small ACCESS-OM2-025}
\vspace{5ex}
\end{flushright}
\end{minipage}%
\hfill
\begin{minipage}[c]{0.8\textwidth}
\iceplotnocbar[.49\textwidth]{#1}{#2}{#3}{#7}\hfill%
\iceplotnocbar[.49\textwidth]{#1}{#2}{#4}{#7}\\
\iceplotnocbar[.49\textwidth]{#1}{#2}{#5}{#7}\hfill%
\iceplotnocbar[.49\textwidth]{#1}{#2}{#6}{#7}\\
\phantom{\quad}\hfill\icecbar[0.8\textwidth]{#1}{#2}{#3}{#7}\hfill\phantom{\quad}%
\end{minipage}%
\hfill
\begin{minipage}[c]{0.09\textwidth}
\begin{flushleft}
\textbf{1$^\circ$}\\
%{\small ACCESS-OM2}\\
\vspace{5ex}
\textbf{0.1$^\circ$}\\
%{\small ACCESS-OM2-01}
\vspace{5ex}
\end{flushleft}
\end{minipage}
}}

\renewcommand{\years}{}

\newcommand{\ABaice}[1]{% cycle 1 at 3 resolutions
\sicfourpanelsAB{Amundsen-Bellingshausen}{aice}{obs}{1deg_jra55_iaf_omip2-fixed_cycle1}{025deg_jra55_iaf_amoctopo_cycle1}{01deg_jra55v140_iaf_cycle1}{#1}{SIC}}
\ABaice{2017-01-01}
%\ABaice{2017-01-02}
%\ABaice{2017-01-03}
%\ABaice{2017-01-04}
\ABaice{2017-01-05}
%\ABaice{2017-01-06}
%\ABaice{2017-01-07}
%\ABaice{2017-01-08}
\ABaice{2017-01-09}
%\ABaice{2017-01-10}
%\ABaice{2017-01-11}
%\ABaice{2017-01-12}
\ABaice{2017-01-13}
%\ABaice{2017-01-14}
%\ABaice{2017-01-15}
%\ABaice{2017-01-16}
\ABaice{2017-01-17}
%\ABaice{2017-01-18}
%\ABaice{2017-01-19}
%\ABaice{2017-01-20}
\ABaice{2017-01-21}
%\ABaice{2017-01-22}

\frame{\frametitle{3 cycles of the 0.1$^\circ$ experiment}
\begin{itemize}
\item Nearly identical in the 3 cycles apart from smallest scales due to uncorrelated eddies
\end{itemize}
}
\newcommand{\sicfourpanelsABcyc}[8]{% #1=view, #2=variable, #3=experiment1, #4=experiment2, #5=experiment3, #6=experiment4, #7=date, #8=title
\frame{
\frametitle{#8\\{\small #7}}
\vspace{-7ex}
\begin{minipage}[c]{0.09\textwidth}
\begin{flushright}
\textbf{Obs}\\
%\textbf{Observational estimate}\\
%{\small (NOAA passive microwave)}\\
\vspace{5ex}
\textbf{Cycle~2}\\
%{\small ACCESS-OM2-025}
\vspace{5ex}
\end{flushright}
\end{minipage}%
\hfill
\begin{minipage}[c]{0.8\textwidth}
\iceplotnocbar[.49\textwidth]{#1}{#2}{#3}{#7}\hfill%
\iceplotnocbar[.49\textwidth]{#1}{#2}{#4}{#7}\\
\iceplotnocbar[.49\textwidth]{#1}{#2}{#5}{#7}\hfill%
\iceplotnocbar[.49\textwidth]{#1}{#2}{#6}{#7}\\
\phantom{\quad}\hfill\icecbar[0.8\textwidth]{#1}{#2}{#3}{#7}\hfill\phantom{\quad}%
\end{minipage}%
\hfill
\begin{minipage}[c]{0.09\textwidth}
\begin{flushleft}
\textbf{Cycle~1}\\
%{\small ACCESS-OM2}\\
\vspace{5ex}
\textbf{Cycle~3}\\
%{\small ACCESS-OM2-01}
\vspace{5ex}
\end{flushleft}
\end{minipage}
}}


\renewcommand{\ABaice}[1]{% now look at 3 cycles of the 0.1deg
%\sicfourpanels[.4\textwidth]{Amundsen-Bellingshausen}{aice}{obs}{01deg_jra55v140_iaf_cycle1}{01deg_jra55v140_iaf_cycle2}{01deg_jra55v140_iaf_cycle3}{#1}}
\sicfourpanelsABcyc{Amundsen-Bellingshausen}{aice}{obs}{01deg_jra55v140_iaf_cycle1}{01deg_jra55v140_iaf_cycle2}{01deg_jra55v140_iaf_cycle3}{#1}{SIC}}
\ABaice{2017-01-01}
%\ABaice{2017-01-02}
%\ABaice{2017-01-03}
%\ABaice{2017-01-04}
\ABaice{2017-01-05}
%\ABaice{2017-01-06}
%\ABaice{2017-01-07}
%\ABaice{2017-01-08}
\ABaice{2017-01-09}
%\ABaice{2017-01-10}
%\ABaice{2017-01-11}
%\ABaice{2017-01-12}
\ABaice{2017-01-13}
%\ABaice{2017-01-14}
%\ABaice{2017-01-15}
%\ABaice{2017-01-16}
\ABaice{2017-01-17}
%\ABaice{2017-01-18}
%\ABaice{2017-01-19}
%\ABaice{2017-01-20}
\ABaice{2017-01-21}
%\ABaice{2017-01-22}
\renewcommand{\years}{\monthlymean}



\section{Perturbation experiments at 1$^\circ$ and 0.25$^\circ$}

%\subsection{Overview}

\frame{
\frametitle{Parameter perturbations at 1$^\circ$ and 0.25$^\circ$}
\begin{itemize}
%\item Perturbation experiments are in \texttt{/home/156/aek156/payu/param_ensemble/access-om2}
%\item The control configuration uses the same parameters as \url{https://github.com/COSIMA/1deg_jra55_iaf/tree/985eebd} (apart from diagnostics)
%\item The members of the perturbation ensemble are defined in \url{https://github.com/aekiss/ensemble/blob/1deg_param_ensemble/ensemble.yaml}
\item Perturbations from 1 January 1968 in the 3rd cycle %of Hakase's OMIP2 experiment \texttt{/scratch/v45/hh0162/access-om2/control/1deg_jra55_iaf_omip2-fixed}, 
and run to the end of 2018 (51 years).
\item Only one parameter varied at a time.
\end{itemize}
}


\frame{
\frametitle{Parameter perturbations at 1$^\circ$ (control values in \textbf{bold}; 0.25$^\circ$ in \textcolor{red}{red})}
\begin{itemize}
\item    ice/cice_in.nml:
\begin{itemize}
\item        shortwave_nml:
\begin{itemize}
\item            albicei: [ 0.36, 0.39, \textbf{0.44}, 0.47 ] (ice IR albedo)
\item            albicev: [ 0.78, 0.81, \textbf{0.86}, \textcolor{red}{0.90} ] (ice visible albedo)
\item            snowpatch: [ 0.005, 0.01, \textbf{0.02} ] (snow patchiness)
\end{itemize}
\end{itemize}
\begin{itemize}
\item        thermo_nml:
\begin{itemize}
\item            chio: [ 0.001, 0.002, 0.003, 0.004, \textbf{0.006} ] (ice-ocean heat transfer coeff)%# see Uotila et al 2012
\end{itemize}
\end{itemize}
\begin{itemize}
\item        dynamics_nml:
\begin{itemize}
\item            turning_angle: [ \textbf{0}, 1, 2, \ldots %3, 4, 5, 6, 7, 8, 
\textcolor{red}{5}, \ldots
9, 10, 12, 14, \ldots %16, 18, 20, 22, 
24, 26 ] (ocean Ek layer)
\item            dragio: [ \textcolor{red}{0.004}, \textbf{0.00536}] (ice-ocean drag coeff)
\item            mu_rdg: [ 2, \textbf{3}, 4, \textcolor{red}{5} ] (ridging scale)
\item            ndte: [ \textbf{120}, 240 ] (elastic sub-timesteps)
%#            revised_evp: [ true, false ]  # true requires a shorter timestep (at least at first)
\end{itemize}
\end{itemize}
\end{itemize}
\begin{itemize}
\item    ocean/input.nml:
%#        auscom_ice_nml:
%#            aice_cutoff: [ ]
\begin{itemize}
\item        ocean_vert_mix_nml:
\begin{itemize}
\item            j09_bgmax: [ 1.e-6, \textbf{5.e-6}, 1.e-5 ] (ocean vertical diffusivity)
\end{itemize}
\end{itemize}
\end{itemize}
}


%\frame{
%\frametitle{Parameter perturbations at 1$^\circ$ and 0.25$^\circ$}
%\begin{itemize}
%\item Perturbation experiments are in \texttt{/home/156/aek156/payu/param_ensemble/access-om2}
%\item The control configuration uses the same parameters as \url{https://github.com/COSIMA/1deg_jra55_iaf/tree/985eebd} (apart from diagnostics)
%\item The members of the perturbation ensemble are defined in \url{https://github.com/aekiss/ensemble/blob/1deg_param_ensemble/ensemble.yaml}
%\item Perturbations and control start from 1 January 1968 in the 3rd cycle of Hakase's OMIP2 experiment \texttt{/scratch/v45/hh0162/access-om2/control/1deg_jra55_iaf_omip2-fixed}, and run to the end of 2018 (51 years).
%\end{itemize}
%}

\subsection{turning angle}

\frame{
\frametitle{turning_angle: parameterisation of unresolved Ekman layer}
\includegraphics[width=\textwidth]{figs/ice_area_min_mean_max_turning_angle.pdf}
}

\frame{
\frametitle{Sea ice mass tendency terms}
%\includegraphics[width=.49\textwidth]{figs/SH_1deg_jra55_iaf_ensemble_turning_angle_5_cycle1_1993-2017_mass_terms_seasonal_clim.pdf}
\includegraphics[width=.49\textwidth]{figs/SH_1deg_jra55_iaf_ensemble_turning_angle_16_cycle1_1993-2017_mass_terms_seasonal_clim.pdf}
\includegraphics[width=.49\textwidth]{figs/SH_1deg_jra55_iaf_ensemble_turning_angle_26_cycle1_1993-2017_mass_terms_seasonal_clim.pdf}\\
dashed is control
}

\renewcommand{\pwidth}{0.24\textwidth}
\renewcommand{\plotter}[2]{% #1=view, #2=date
\frame{
\plotnocbar[\pwidth]{figs/#1_aice_m_mm_1deg_jra55_iaf_omip2_cycle3-obs_1993-2017_mean_month_#2_200dpi.png}%
\plotnocbar[\pwidth]{figs/#1_aice_m_mm_1deg_jra55_iaf_ensemble_turning_angle_5_-control_1993-2017_mean_month_#2_200dpi.png}%
\plotnocbar[\pwidth]{figs/#1_aice_m_mm_1deg_jra55_iaf_ensemble_turning_angle_16_-control_1993-2017_mean_month_#2_200dpi.png}%
\plotnocbar[\pwidth]{figs/#1_aice_m_mm_1deg_jra55_iaf_ensemble_turning_angle_26_-control_1993-2017_mean_month_#2_200dpi.png}\\
\plotnocbar[\pwidth]{figs/#1_aice_m_mm_025deg_jra55_iaf_omip2_cycle3-obs_1993-2017_mean_month_#2_200dpi.png}%
\plotnocbar[\pwidth]{figs/#1_aice_m_mm_025deg_jra55_iaf_ensemble_turning_angle_5_-control_1993-2017_mean_month_#2_200dpi.png}\\
\cbar[\pwidth]{figs/#1_aice_m_mm_1deg_jra55_iaf_omip2_cycle3-obs_1993-2017_mean_month_#2_200dpi.png}%
\cbar[0.5\textwidth]{figs/#1_aice_m_mm_025deg_jra55_iaf_ensemble_turning_angle_5_-control_1993-2017_mean_month_#2_200dpi.png}
}}
\foreach \m in {01,02,03,04,05,06,07,08,09,10,11,12}{\plotter{SH}{\m}}
%\foreach \m in {01,03,05,07,09,11}{\plotter{SH}{\m}}

\renewcommand{\pwidth}{0.24\textwidth}
\renewcommand{\plotter}[2]{% #1=view, #2=date
\frame{
\plotnocbar[\pwidth]{figs/#1_aice_m_mm_1deg_jra55_iaf_omip2_cycle3-obs_1993-2017_mean_month_#2_200dpi.png}%
\plotnocbar[\pwidth]{figs/#1_aice_m_mm_1deg_jra55_iaf_ensemble_turning_angle_5_cycle1-obs_1993-2017_mean_month_#2_200dpi.png}%
\plotnocbar[\pwidth]{figs/#1_aice_m_mm_1deg_jra55_iaf_ensemble_turning_angle_16_cycle1-obs_1993-2017_mean_month_#2_200dpi.png}%
\plotnocbar[\pwidth]{figs/#1_aice_m_mm_1deg_jra55_iaf_ensemble_turning_angle_26_cycle1-obs_1993-2017_mean_month_#2_200dpi.png}\\
\plotnocbar[\pwidth]{figs/#1_aice_m_mm_025deg_jra55_iaf_omip2_cycle3-obs_1993-2017_mean_month_#2_200dpi.png}%
\plotnocbar[\pwidth]{figs/#1_aice_m_mm_025deg_jra55_iaf_ensemble_turning_angle_5_cycle1-obs_1993-2017_mean_month_#2_200dpi.png}\\
\cbar[0.5\textwidth]{figs/#1_aice_m_mm_1deg_jra55_iaf_omip2_cycle3-obs_1993-2017_mean_month_#2_200dpi.png}%
}}
\foreach \m in {01,02,03,04,05,06,07,08,09,10,11,12}{\plotter{SH}{\m}}
%\foreach \m in {01,03,05,07,09,11}{\plotter{SH}{\m}}

%\sicsixpanelspertcompare{SH}{aice_m_mm_1deg_jra55_iaf_ensemble}{_turning_angle_5}{_turning_angle_26}{}{_turning_angle_5_-control}{_turning_angle_26_-control}{01}
%\sicsixpanelspertcompare{SH}{aice_m_mm_1deg_jra55_iaf_ensemble}{_turning_angle_5}{_turning_angle_26}{}{_turning_angle_5_-control}{_turning_angle_26_-control}{03}
%\sicsixpanelspertcompare{SH}{aice_m_mm_1deg_jra55_iaf_ensemble}{_turning_angle_5}{_turning_angle_26}{}{_turning_angle_5_-control}{_turning_angle_26_-control}{05}
%\sicsixpanelspertcompare{SH}{aice_m_mm_1deg_jra55_iaf_ensemble}{_turning_angle_5}{_turning_angle_26}{}{_turning_angle_5_-control}{_turning_angle_26_-control}{07}
%\sicsixpanelspertcompare{SH}{aice_m_mm_1deg_jra55_iaf_ensemble}{_turning_angle_5}{_turning_angle_26}{}{_turning_angle_5_-control}{_turning_angle_26_-control}{09}
%\sicsixpanelspertcompare{SH}{aice_m_mm_1deg_jra55_iaf_ensemble}{_turning_angle_5}{_turning_angle_26}{}{_turning_angle_5_-control}{_turning_angle_26_-control}{11}


%\frame{
%\frametitle{turning_angle: parameterisation of unresolved Ekman layer}
%\includegraphics[width=\textwidth]{figs/ice_area_min_mean_max_turning_angle.pdf}
%}
%\sicsixpanelspertcompare{SH}{aice_m_mm_1deg_jra55_iaf_ensemble}{_turning_angle_5}{_turning_angle_5}{}{_turning_angle_5_-control}{_turning_angle_5_-control}{01}
%\sicsixpanelspertcompare{SH}{aice_m_mm_025deg_jra55_iaf_ensemble}{_turning_angle_5}{_turning_angle_5}{}{_turning_angle_5_-control}{_turning_angle_5_-control}{01}
%\sicsixpanelspertcompare{SH}{aice_m_mm_1deg_jra55_iaf_ensemble}{_turning_angle_5}{_turning_angle_5}{}{_turning_angle_5_-control}{_turning_angle_5_-control}{09}
%\sicsixpanelspertcompare{SH}{aice_m_mm_025deg_jra55_iaf_ensemble}{_turning_angle_5}{_turning_angle_5}{}{_turning_angle_5_-control}{_turning_angle_5_-control}{09}

\frame{
\frametitle{turning_angle summary}
\begin{itemize}
\item Insignificant impact on overall bias
\item Little dependence on resolution
\item Perturbation pattern changes only in amplitude as turning_angle increases
\item Coastal SIC increases with turning_angle
\item SIC decreases with turning_angle further offshore
\item SIC increases with turning_angle in outermost region in spring
\item Cancellation reduces signal in total area timeseries
\end{itemize}
}

\subsection{dragio and ndte}

\frame{
\frametitle{dragio: ice-ocean drag coefficient}
\includegraphics[width=\textwidth]{figs/ice_area_min_mean_max_dragio.pdf}
}

\frame{
\frametitle{ndte: number of elastic sub-timesteps}
\includegraphics[width=\textwidth]{figs/ice_area_min_mean_max_ndte.pdf}
}

\frame{
\frametitle{Sea ice mass tendency terms}
\includegraphics[width=.49\textwidth]{{figs/SH_1deg_jra55_iaf_ensemble_dragio_0.004_cycle1_1993-2017_mass_terms_seasonal_clim}.pdf}
\includegraphics[width=.49\textwidth]{figs/SH_1deg_jra55_iaf_ensemble_ndte_240_cycle1_1993-2017_mass_terms_seasonal_clim.pdf}\\
dashed is control
}

\renewcommand{\pwidth}{0.24\textwidth}
\renewcommand{\plotter}[2]{% #1=view, #2=date
\frame{
\plotnocbar[\pwidth]{figs/#1_aice_m_mm_1deg_jra55_iaf_omip2_cycle3-obs_1993-2017_mean_month_#2_200dpi.png}%
\plotnocbar[\pwidth]{{figs/#1_aice_m_mm_1deg_jra55_iaf_ensemble_dragio_0.004_-control_1993-2017_mean_month_#2_200dpi}.png}%
\plotnocbar[\pwidth]{{figs/#1_aice_m_mm_1deg_jra55_iaf_ensemble_ndte_240_-control_1993-2017_mean_month_#2_200dpi}.png}\\
\plotnocbar[\pwidth]{figs/#1_aice_m_mm_025deg_jra55_iaf_omip2_cycle3-obs_1993-2017_mean_month_#2_200dpi.png}%
\plotnocbar[\pwidth]{{figs/#1_aice_m_mm_025deg_jra55_iaf_ensemble_dragio_0.004_-control_1993-2017_mean_month_#2_200dpi}.png}\\
\cbar[\pwidth]{figs/#1_aice_m_mm_1deg_jra55_iaf_omip2_cycle3-obs_1993-2017_mean_month_#2_200dpi.png}%
\cbar[0.5\textwidth]{{figs/#1_aice_m_mm_025deg_jra55_iaf_ensemble_dragio_0.004_-control_1993-2017_mean_month_#2_200dpi}.png}
}}
\foreach \m in {01,02,03,04,05,06,07,08,09,10,11,12}{\plotter{SH}{\m}}
%\foreach \m in {01,03,05,07,09,11}{\plotter{SH}{\m}}

\renewcommand{\pwidth}{0.24\textwidth}
\renewcommand{\plotter}[2]{% #1=view, #2=date
\frame{
\plotnocbar[\pwidth]{figs/#1_aice_m_mm_1deg_jra55_iaf_omip2_cycle3-obs_1993-2017_mean_month_#2_200dpi.png}%
\plotnocbar[\pwidth]{{figs/#1_aice_m_mm_1deg_jra55_iaf_ensemble_dragio_0.004_cycle1-obs_1993-2017_mean_month_#2_200dpi}.png}%
\plotnocbar[\pwidth]{{figs/#1_aice_m_mm_1deg_jra55_iaf_ensemble_ndte_240_cycle1-obs_1993-2017_mean_month_#2_200dpi}.png}\\
\plotnocbar[\pwidth]{figs/#1_aice_m_mm_025deg_jra55_iaf_omip2_cycle3-obs_1993-2017_mean_month_#2_200dpi.png}%
\plotnocbar[\pwidth]{{figs/#1_aice_m_mm_025deg_jra55_iaf_ensemble_dragio_0.004_cycle1-obs_1993-2017_mean_month_#2_200dpi}.png}\\
\cbar[0.5\textwidth]{figs/#1_aice_m_mm_1deg_jra55_iaf_omip2_cycle3-obs_1993-2017_mean_month_#2_200dpi.png}%
}}
\foreach \m in {01,02,03,04,05,06,07,08,09,10,11,12}{\plotter{SH}{\m}}
%\foreach \m in {01,03,05,07,09,11}{\plotter{SH}{\m}}

%%\frame{
%%\frametitle{dragio}
%%\includegraphics[width=\textwidth]{figs/ice_area_min_mean_max_mu_rdg.pdf}
%%}
%\sicsixpanelspertcompare{SH}{aice_m_mm_1deg_jra55_iaf_ensemble}{_dragio_0.004}{_dragio_0.004}{}{_dragio_0.004_-control}{_dragio_0.004_-control}{01}
%\sicsixpanelspertcompare{SH}{aice_m_mm_025deg_jra55_iaf_ensemble}{_dragio_0.004}{_dragio_0.004}{}{_dragio_0.004_-control}{_dragio_0.004_-control}{01}
%\sicsixpanelspertcompare{SH}{aice_m_mm_1deg_jra55_iaf_ensemble}{_dragio_0.004}{_dragio_0.004}{}{_dragio_0.004_-control}{_dragio_0.004_-control}{09}
%\sicsixpanelspertcompare{SH}{aice_m_mm_025deg_jra55_iaf_ensemble}{_dragio_0.004}{_dragio_0.004}{}{_dragio_0.004_-control}{_dragio_0.004_-control}{09}

%\sicsixpanelspertcompare{SH}{aice_m_mm_1deg_jra55_iaf_ensemble}{_dragio_0.004}{_ndte_240}{}{_dragio_0.004_-control}{_ndte_240_-control}{01}
%\sicsixpanelspertcompare{SH}{aice_m_mm_1deg_jra55_iaf_ensemble}{_dragio_0.004}{_ndte_240}{}{_dragio_0.004_-control}{_ndte_240_-control}{03}
%\sicsixpanelspertcompare{SH}{aice_m_mm_1deg_jra55_iaf_ensemble}{_dragio_0.004}{_ndte_240}{}{_dragio_0.004_-control}{_ndte_240_-control}{05}
%\sicsixpanelspertcompare{SH}{aice_m_mm_1deg_jra55_iaf_ensemble}{_dragio_0.004}{_ndte_240}{}{_dragio_0.004_-control}{_ndte_240_-control}{07}
%\sicsixpanelspertcompare{SH}{aice_m_mm_1deg_jra55_iaf_ensemble}{_dragio_0.004}{_ndte_240}{}{_dragio_0.004_-control}{_ndte_240_-control}{09}
%\sicsixpanelspertcompare{SH}{aice_m_mm_1deg_jra55_iaf_ensemble}{_dragio_0.004}{_ndte_240}{}{_dragio_0.004_-control}{_ndte_240_-control}{11}
\frame{
\frametitle{dragio and ndte summary}
\begin{itemize}
\item Neither dragio nor ndte make much difference to bias
\item Little dependence on resolution
\item Reducing dragio from its control value (0.00536) generally decreases SIC in summer and autumn; patterns are more complex in other seasons, with cancellation between positive and negative regions
\item Doubling ndte from its control value (120) has greatest effect in summer, where it increases SIC. It has little effect in other seasons except for patchy changes in the MIZ. This generally weak dependence on ndte indicates near-convergence of the EVP solver.
\end{itemize}
}

\subsection{mu_rdg}

\frame{
\frametitle{mu_rdg: ridging parameter}
\includegraphics[width=\textwidth]{figs/ice_area_min_mean_max_mu_rdg.pdf}
}

\frame{
\frametitle{Sea ice mass tendency terms}
\includegraphics[width=.49\textwidth]{figs/SH_1deg_jra55_iaf_ensemble_mu_rdg_5_cycle1_1993-2017_mass_terms_seasonal_clim.pdf}
\includegraphics[width=.49\textwidth]{figs/SH_1deg_jra55_iaf_ensemble_mu_rdg_2_cycle1_1993-2017_mass_terms_seasonal_clim.pdf}\\
dashed is control
}

\renewcommand{\pwidth}{0.24\textwidth}
\renewcommand{\plotter}[2]{% #1=view, #2=date
\frame{
\plotnocbar[\pwidth]{figs/#1_aice_m_mm_1deg_jra55_iaf_omip2_cycle3-obs_1993-2017_mean_month_#2_200dpi.png}%
\plotnocbar[\pwidth]{{figs/#1_aice_m_mm_1deg_jra55_iaf_ensemble_mu_rdg_5_-control_1993-2017_mean_month_#2_200dpi}.png}%
\plotnocbar[\pwidth]{{figs/#1_aice_m_mm_1deg_jra55_iaf_ensemble_mu_rdg_4_-control_1993-2017_mean_month_#2_200dpi}.png}%
\plotnocbar[\pwidth]{{figs/#1_aice_m_mm_1deg_jra55_iaf_ensemble_mu_rdg_2_-control_1993-2017_mean_month_#2_200dpi}.png}\\
\plotnocbar[\pwidth]{figs/#1_aice_m_mm_025deg_jra55_iaf_omip2_cycle3-obs_1993-2017_mean_month_#2_200dpi.png}%
\plotnocbar[\pwidth]{{figs/#1_aice_m_mm_025deg_jra55_iaf_ensemble_mu_rdg_5_-control_1993-2017_mean_month_#2_200dpi}.png}\\%
\cbar[\pwidth]{figs/#1_aice_m_mm_1deg_jra55_iaf_omip2_cycle3-obs_1993-2017_mean_month_#2_200dpi.png}%
\cbar[0.5\textwidth]{{figs/#1_aice_m_mm_1deg_jra55_iaf_ensemble_mu_rdg_5_-control_1993-2017_mean_month_#2_200dpi}.png}%
}}
\foreach \m in {01,02,03,04,05,06,07,08,09,10,11,12}{\plotter{SH}{\m}}
%\foreach \m in {01,03,05,07,09,11}{\plotter{SH}{\m}}

\renewcommand{\pwidth}{0.24\textwidth}
\renewcommand{\plotter}[2]{% #1=view, #2=date
\frame{
\plotnocbar[\pwidth]{figs/#1_aice_m_mm_1deg_jra55_iaf_omip2_cycle3-obs_1993-2017_mean_month_#2_200dpi.png}%
\plotnocbar[\pwidth]{{figs/#1_aice_m_mm_1deg_jra55_iaf_ensemble_mu_rdg_5_cycle1-obs_1993-2017_mean_month_#2_200dpi}.png}%
\plotnocbar[\pwidth]{{figs/#1_aice_m_mm_1deg_jra55_iaf_ensemble_mu_rdg_4_cycle1-obs_1993-2017_mean_month_#2_200dpi}.png}%
\plotnocbar[\pwidth]{{figs/#1_aice_m_mm_1deg_jra55_iaf_ensemble_mu_rdg_2_cycle1-obs_1993-2017_mean_month_#2_200dpi}.png}\\
\plotnocbar[\pwidth]{figs/#1_aice_m_mm_025deg_jra55_iaf_omip2_cycle3-obs_1993-2017_mean_month_#2_200dpi.png}%
\plotnocbar[\pwidth]{{figs/#1_aice_m_mm_025deg_jra55_iaf_ensemble_mu_rdg_5_cycle1-obs_1993-2017_mean_month_#2_200dpi}.png}\\
\cbar[0.5\textwidth]{figs/#1_aice_m_mm_1deg_jra55_iaf_omip2_cycle3-obs_1993-2017_mean_month_#2_200dpi.png}%
}}
\foreach \m in {01,02,03,04,05,06,07,08,09,10,11,12}{\plotter{SH}{\m}}
%\foreach \m in {01,03,05,07,09,11}{\plotter{SH}{\m}}


%\frame{
%\frametitle{mu_rdg}
%\includegraphics[width=\textwidth]{figs/ice_area_min_mean_max_mu_rdg.pdf}
%}
%\sicsixpanelspertcompare{SH}{aice_m_mm_1deg_jra55_iaf_ensemble}{_mu_rdg_5}{_mu_rdg_5}{}{_mu_rdg_5_-control}{_mu_rdg_5_-control}{01}
%\sicsixpanelspertcompare{SH}{aice_m_mm_025deg_jra55_iaf_ensemble}{_mu_rdg_5}{_mu_rdg_5}{}{_mu_rdg_5_-control}{_mu_rdg_5_-control}{01}
%\sicsixpanelspertcompare{SH}{aice_m_mm_1deg_jra55_iaf_ensemble}{_mu_rdg_5}{_mu_rdg_5}{}{_mu_rdg_5_-control}{_mu_rdg_5_-control}{09}
%\sicsixpanelspertcompare{SH}{aice_m_mm_025deg_jra55_iaf_ensemble}{_mu_rdg_5}{_mu_rdg_5}{}{_mu_rdg_5_-control}{_mu_rdg_5_-control}{09}
%\sicsixpanelspertcompare{SH}{aice_m_mm_1deg_jra55_iaf_ensemble}{_mu_rdg_2}{_mu_rdg_5}{}{_mu_rdg_2_-control}{_mu_rdg_5_-control}{01}
%\sicsixpanelspertcompare{SH}{aice_m_mm_1deg_jra55_iaf_ensemble}{_mu_rdg_2}{_mu_rdg_5}{}{_mu_rdg_2_-control}{_mu_rdg_5_-control}{03}
%\sicsixpanelspertcompare{SH}{aice_m_mm_1deg_jra55_iaf_ensemble}{_mu_rdg_2}{_mu_rdg_5}{}{_mu_rdg_2_-control}{_mu_rdg_5_-control}{05}
%\sicsixpanelspertcompare{SH}{aice_m_mm_1deg_jra55_iaf_ensemble}{_mu_rdg_2}{_mu_rdg_5}{}{_mu_rdg_2_-control}{_mu_rdg_5_-control}{07}
%\sicsixpanelspertcompare{SH}{aice_m_mm_1deg_jra55_iaf_ensemble}{_mu_rdg_2}{_mu_rdg_5}{}{_mu_rdg_2_-control}{_mu_rdg_5_-control}{09}
%\sicsixpanelspertcompare{SH}{aice_m_mm_1deg_jra55_iaf_ensemble}{_mu_rdg_2}{_mu_rdg_5}{}{_mu_rdg_2_-control}{_mu_rdg_5_-control}{11}

\frame{
\frametitle{mu_rdg summary}
\begin{itemize}
\item Impact on bias is weak
\item Little dependence on resolution
\item Perturbation pattern changes only in amplitude as mu_rdg increases
\item Perturbation pattern has uniform sign in summer and early autumn, with larger (smaller) SIC for mu_rdg smaller (larger) than control value (3)
\item Perturbation pattern is much weaker and more complex in other seasons; cancellation reduces signal in total area timeseries
\end{itemize}
}

\subsection{j09_bgmax}

\frame{
\frametitle{j09_bgmax: non-equatorial background vertical diffusivity}
\includegraphics[width=\textwidth]{figs/ice_area_min_mean_max_j09_bgmax.pdf}
}

\frame{
\frametitle{Sea ice mass tendency terms}
\includegraphics[width=.49\textwidth]{figs/SH_1deg_jra55_iaf_ensemble_j09_bgmax_1e-06_cycle1_1993-2017_mass_terms_seasonal_clim.pdf}
\includegraphics[width=.49\textwidth]{figs/SH_1deg_jra55_iaf_ensemble_j09_bgmax_1e-05_cycle1_1993-2017_mass_terms_seasonal_clim.pdf}\\
dashed is control
}

\renewcommand{\pwidth}{0.3\textwidth}
\renewcommand{\plotter}[2]{% #1=view, #2=date
\frame{
\plotnocbar[\pwidth]{figs/#1_aice_m_mm_1deg_jra55_iaf_omip2_cycle3-obs_1993-2017_mean_month_#2_200dpi.png}%
\plotnocbar[\pwidth]{{figs/#1_aice_m_mm_1deg_jra55_iaf_ensemble_j09_bgmax_1e-06_cycle1-obs_1993-2017_mean_month_#2_200dpi}.png}%
\plotnocbar[\pwidth]{{figs/#1_aice_m_mm_1deg_jra55_iaf_ensemble_j09_bgmax_1e-05_cycle1-obs_1993-2017_mean_month_#2_200dpi}.png}%
\cbarrot[\pwidth]{figs/#1_aice_m_mm_1deg_jra55_iaf_omip2_cycle3-obs_1993-2017_mean_month_#2_200dpi.png}\\%
\hspace{\pwidth}%
\plotnocbar[\pwidth]{{figs/#1_aice_m_mm_1deg_jra55_iaf_ensemble_j09_bgmax_1e-06_-control_1993-2017_mean_month_#2_200dpi}.png}%
\plotnocbar[\pwidth]{{figs/#1_aice_m_mm_1deg_jra55_iaf_ensemble_j09_bgmax_1e-05_-control_1993-2017_mean_month_#2_200dpi}.png}%
\cbarrot[\pwidth]{{figs/#1_aice_m_mm_1deg_jra55_iaf_ensemble_j09_bgmax_1e-06_-control_1993-2017_mean_month_#2_200dpi}.png}%
}}
\foreach \m in {01,02,03,04,05,06,07,08,09,10,11,12}{\plotter{SH}{\m}}
%\foreach \m in {01,03,05,07,09,11}{\plotter{SH}{\m}}
%\sicsixpanelspertcompare{SH}{aice_m_mm_1deg_jra55_iaf_ensemble}{_j09_bgmax_1e-06}{_j09_bgmax_1e-05}{}{_j09_bgmax_1e-06_-control}{_j09_bgmax_1e-05_-control}{01}
%\sicsixpanelspertcompare{SH}{aice_m_mm_1deg_jra55_iaf_ensemble}{_j09_bgmax_1e-06}{_j09_bgmax_1e-05}{}{_j09_bgmax_1e-06_-control}{_j09_bgmax_1e-05_-control}{03}
%\sicsixpanelspertcompare{SH}{aice_m_mm_1deg_jra55_iaf_ensemble}{_j09_bgmax_1e-06}{_j09_bgmax_1e-05}{}{_j09_bgmax_1e-06_-control}{_j09_bgmax_1e-05_-control}{05}
%\sicsixpanelspertcompare{SH}{aice_m_mm_1deg_jra55_iaf_ensemble}{_j09_bgmax_1e-06}{_j09_bgmax_1e-05}{}{_j09_bgmax_1e-06_-control}{_j09_bgmax_1e-05_-control}{07}
%\sicsixpanelspertcompare{SH}{aice_m_mm_1deg_jra55_iaf_ensemble}{_j09_bgmax_1e-06}{_j09_bgmax_1e-05}{}{_j09_bgmax_1e-06_-control}{_j09_bgmax_1e-05_-control}{09}
%\sicsixpanelspertcompare{SH}{aice_m_mm_1deg_jra55_iaf_ensemble}{_j09_bgmax_1e-06}{_j09_bgmax_1e-05}{}{_j09_bgmax_1e-06_-control}{_j09_bgmax_1e-05_-control}{11}
\frame{
\frametitle{j09_bgmax summary}
\begin{itemize}
\item Parameter has negligible effect on SIC over range tested
\item Perturbation pattern changes mainly in amplitude as j09_bgmax increases
\item Perturbation pattern cancellation reduces signal in total area timeseries
\end{itemize}
}

\subsection{chio}

\frame{
\frametitle{chio: coefficient of ice-ocean heat flux}
\includegraphics[width=\textwidth]{figs/ice_area_min_mean_max_chio.pdf}
}

\frame{
\frametitle{Sea ice mass tendency terms}
\includegraphics[width=.49\textwidth]{{figs/SH_1deg_jra55_iaf_ensemble_chio_0.003_cycle1_1993-2017_mass_terms_seasonal_clim}.pdf}
\includegraphics[width=.49\textwidth]{{figs/SH_1deg_jra55_iaf_ensemble_chio_0.001_cycle1_1993-2017_mass_terms_seasonal_clim}.pdf}\\
dashed is control
}

\renewcommand{\pwidth}{0.3\textwidth}
\renewcommand{\plotter}[2]{% #1=view, #2=date
\frame{
\plotnocbar[\pwidth]{figs/#1_aice_m_mm_1deg_jra55_iaf_omip2_cycle3-obs_1993-2017_mean_month_#2_200dpi.png}%
%\plotnocbar[\pwidth]{{figs/#1_aice_m_mm_1deg_jra55_iaf_ensemble_chio_0.004_cycle1-obs_1993-2017_mean_month_#2_200dpi}.png}%
\plotnocbar[\pwidth]{{figs/#1_aice_m_mm_1deg_jra55_iaf_ensemble_chio_0.003_cycle1-obs_1993-2017_mean_month_#2_200dpi}.png}%
%\plotnocbar[\pwidth]{{figs/#1_aice_m_mm_1deg_jra55_iaf_ensemble_chio_0.002_cycle1-obs_1993-2017_mean_month_#2_200dpi}.png}%
\plotnocbar[\pwidth]{{figs/#1_aice_m_mm_1deg_jra55_iaf_ensemble_chio_0.001_cycle1-obs_1993-2017_mean_month_#2_200dpi}.png}%
\cbarrot[\pwidth]{{figs/#1_aice_m_mm_1deg_jra55_iaf_ensemble_chio_0.004_cycle1-obs_1993-2017_mean_month_#2_200dpi}.png}\\%
\hspace{\pwidth}%
%\plotnocbar[\pwidth]{{figs/#1_aice_m_mm_1deg_jra55_iaf_ensemble_chio_0.004_-control_1993-2017_mean_month_#2_200dpi}.png}%
\plotnocbar[\pwidth]{{figs/#1_aice_m_mm_1deg_jra55_iaf_ensemble_chio_0.003_-control_1993-2017_mean_month_#2_200dpi}.png}%
%\plotnocbar[\pwidth]{{figs/#1_aice_m_mm_1deg_jra55_iaf_ensemble_chio_0.002_-control_1993-2017_mean_month_#2_200dpi}.png}%
\plotnocbar[\pwidth]{{figs/#1_aice_m_mm_1deg_jra55_iaf_ensemble_chio_0.001_-control_1993-2017_mean_month_#2_200dpi}.png}%
\cbarrot[\pwidth]{{figs/#1_aice_m_mm_1deg_jra55_iaf_ensemble_chio_0.004_-control_1993-2017_mean_month_#2_200dpi}.png}\\%
}}
\foreach \m in {01,02,03,04,05,06,07,08,09,10,11,12}{\plotter{SH}{\m}}
%\foreach \m in {01,03,05,07,09,11}{\plotter{SH}{\m}}
%\sicsixpanelspertcompare{SH}{aice_m_mm_1deg_jra55_iaf_ensemble}{_chio_0.004}{_chio_0.002}{}{_chio_0.004_-control}{_chio_0.002_-control}{01}
%\sicsixpanelspertcompare{SH}{aice_m_mm_1deg_jra55_iaf_ensemble}{_chio_0.004}{_chio_0.002}{}{_chio_0.004_-control}{_chio_0.002_-control}{03}
%\sicsixpanelspertcompare{SH}{aice_m_mm_1deg_jra55_iaf_ensemble}{_chio_0.004}{_chio_0.002}{}{_chio_0.004_-control}{_chio_0.002_-control}{05}
%\sicsixpanelspertcompare{SH}{aice_m_mm_1deg_jra55_iaf_ensemble}{_chio_0.004}{_chio_0.002}{}{_chio_0.004_-control}{_chio_0.002_-control}{07}
%\sicsixpanelspertcompare{SH}{aice_m_mm_1deg_jra55_iaf_ensemble}{_chio_0.004}{_chio_0.002}{}{_chio_0.004_-control}{_chio_0.002_-control}{09}
%\sicsixpanelspertcompare{SH}{aice_m_mm_1deg_jra55_iaf_ensemble}{_chio_0.004}{_chio_0.002}{}{_chio_0.004_-control}{_chio_0.002_-control}{11}
\frame{
\frametitle{chio summary}
\begin{itemize}
\item Perturbation pattern changes only in amplitude as chio decreases
\item Perturbation pattern has uniform sign in summer, with SIC increasing as chio decreases from control value (0.006)
\item Perturbation pattern is weaker and more complex in other seasons; cancellation reduces signal in total area timeseries
\item Except in summer, there are large areas in which SIC decreases slightly with decreasing chio. 
From p53 of the CICE manual: 
\textit{The net downward heat flux from the ice to the ocean is given by:
$F_\text{bot} =-\rho_wc_wc_hu^{*}(T_w -T_f)$
where $\rho_w$ is the density of seawater, $c_w$ is the specific heat of seawater, $c_h$ is a heat transfer coefficient [chio], $u^{*}$ is the friction velocity, and $T_w$ is the sea surface temperature.}
$T_f$ is the freezing temperature.
So the regions where SIC decreases with decreasing chio are presumably where $F_\text{bot}>0$ i.e.\ heat flows from ice to ocean, i.e.\ $T_w < T_f$.
\end{itemize}
}

\subsection{snowpatch}

\frame{
\frametitle{snowpatch: snow patchiness length scale}
\includegraphics[width=\textwidth]{figs/ice_area_min_mean_max_snowpatch.pdf}
}

\frame{
\frametitle{Sea ice mass tendency terms}
\includegraphics[width=.49\textwidth]{{figs/SH_1deg_jra55_iaf_ensemble_snowpatch_0.01_cycle1_1993-2017_mass_terms_seasonal_clim}.pdf}
\includegraphics[width=.49\textwidth]{{figs/SH_1deg_jra55_iaf_ensemble_snowpatch_0.005_cycle1_1993-2017_mass_terms_seasonal_clim}.pdf}\\
dashed is control
}

\renewcommand{\pwidth}{0.3\textwidth}
\renewcommand{\plotter}[2]{% #1=view, #2=date
\frame{
\plotnocbar[\pwidth]{figs/#1_aice_m_mm_1deg_jra55_iaf_omip2_cycle3-obs_1993-2017_mean_month_#2_200dpi.png}%
\plotnocbar[\pwidth]{{figs/#1_aice_m_mm_1deg_jra55_iaf_ensemble_snowpatch_0.01_cycle1-obs_1993-2017_mean_month_#2_200dpi}.png}%
\plotnocbar[\pwidth]{{figs/#1_aice_m_mm_1deg_jra55_iaf_ensemble_snowpatch_0.005_cycle1-obs_1993-2017_mean_month_#2_200dpi}.png}%
\cbarrot[\pwidth]{{figs/#1_aice_m_mm_1deg_jra55_iaf_ensemble_snowpatch_0.005_cycle1-obs_1993-2017_mean_month_#2_200dpi}.png}\\%
\hspace{\pwidth}%
\plotnocbar[\pwidth]{{figs/#1_aice_m_mm_1deg_jra55_iaf_ensemble_snowpatch_0.01_-control_1993-2017_mean_month_#2_200dpi}.png}%
\plotnocbar[\pwidth]{{figs/#1_aice_m_mm_1deg_jra55_iaf_ensemble_snowpatch_0.005_-control_1993-2017_mean_month_#2_200dpi}.png}%
\cbarrot[\pwidth]{{figs/#1_aice_m_mm_1deg_jra55_iaf_ensemble_snowpatch_0.01_-control_1993-2017_mean_month_#2_200dpi}.png}\\%
}}
\foreach \m in {01,02,03,04,05,06,07,08,09,10,11,12}{\plotter{SH}{\m}}
%\foreach \m in {01,03,05,07,09,11}{\plotter{SH}{\m}}
%\sicsixpanelspertcompare{SH}{aice_m_mm_1deg_jra55_iaf_ensemble}{_snowpatch_0.01}{_snowpatch_0.005}{}{_snowpatch_0.01_-control}{_snowpatch_0.005_-control}{01}
%\sicsixpanelspertcompare{SH}{aice_m_mm_1deg_jra55_iaf_ensemble}{_snowpatch_0.01}{_snowpatch_0.005}{}{_snowpatch_0.01_-control}{_snowpatch_0.005_-control}{03}
%\sicsixpanelspertcompare{SH}{aice_m_mm_1deg_jra55_iaf_ensemble}{_snowpatch_0.01}{_snowpatch_0.005}{}{_snowpatch_0.01_-control}{_snowpatch_0.005_-control}{05}
%\sicsixpanelspertcompare{SH}{aice_m_mm_1deg_jra55_iaf_ensemble}{_snowpatch_0.01}{_snowpatch_0.005}{}{_snowpatch_0.01_-control}{_snowpatch_0.005_-control}{07}
%\sicsixpanelspertcompare{SH}{aice_m_mm_1deg_jra55_iaf_ensemble}{_snowpatch_0.01}{_snowpatch_0.005}{}{_snowpatch_0.01_-control}{_snowpatch_0.005_-control}{09}
%\sicsixpanelspertcompare{SH}{aice_m_mm_1deg_jra55_iaf_ensemble}{_snowpatch_0.01}{_snowpatch_0.005}{}{_snowpatch_0.01_-control}{_snowpatch_0.005_-control}{11}
\frame{
\frametitle{snowpatch summary}
\begin{itemize}
\item snowpatch is used to determine the snow-covered area, which is then used to determine the total albedo from the area-weighted average of the ice and snow albedos. As snowpatch increases, the snow area decreases.
\item Perturbation pattern changes only in amplitude as snowpatch decreases
\item Perturbation pattern has uniform sign in summer and early autumn, with SIC increasing as snowpatch decreases from control value (0.02\,m), i.e.\ as snow albedo is weighted more strongly
\item Perturbation pattern is much weaker and more complex in other seasons; cancellation reduces signal in total area timeseries. 
\item Presumably snowpatch's effect on the visible albedo is unimportant in the polar night relative to its effect on IR albedo
\end{itemize}
}


\subsection{albicev}

\frame{
\frametitle{albicev: ice visible albedo}
\includegraphics[width=\textwidth]{figs/ice_area_min_mean_max_albicev.pdf}
}

\frame{
\frametitle{Sea ice mass tendency terms}
\includegraphics[width=.49\textwidth]{{figs/SH_1deg_jra55_iaf_ensemble_albicev_0.9_cycle1_1993-2017_mass_terms_seasonal_clim}.pdf}
\includegraphics[width=.49\textwidth]{{figs/SH_1deg_jra55_iaf_ensemble_albicev_0.78_cycle1_1993-2017_mass_terms_seasonal_clim}.pdf}\\
dashed is control
}

\renewcommand{\pwidth}{0.3\textwidth}
\renewcommand{\plotter}[2]{% #1=view, #2=date
\frame{
\plotnocbar[\pwidth]{figs/#1_aice_m_mm_1deg_jra55_iaf_omip2_cycle3-obs_1993-2017_mean_month_#2_200dpi.png}%
\plotnocbar[\pwidth]{{figs/#1_aice_m_mm_1deg_jra55_iaf_ensemble_albicev_0.9_cycle1-obs_1993-2017_mean_month_#2_200dpi}.png}%
\plotnocbar[\pwidth]{{figs/#1_aice_m_mm_1deg_jra55_iaf_ensemble_albicev_0.78_cycle1-obs_1993-2017_mean_month_#2_200dpi}.png}%
\cbarrot[\pwidth]{{figs/#1_aice_m_mm_1deg_jra55_iaf_ensemble_albicev_0.9_cycle1-obs_1993-2017_mean_month_#2_200dpi}.png}\\%
\hspace{\pwidth}%
\plotnocbar[\pwidth]{{figs/#1_aice_m_mm_1deg_jra55_iaf_ensemble_albicev_0.9_-control_1993-2017_mean_month_#2_200dpi}.png}%
\plotnocbar[\pwidth]{{figs/#1_aice_m_mm_1deg_jra55_iaf_ensemble_albicev_0.78_-control_1993-2017_mean_month_#2_200dpi}.png}%
\cbarrot[\pwidth]{{figs/#1_aice_m_mm_1deg_jra55_iaf_ensemble_albicev_0.9_-control_1993-2017_mean_month_#2_200dpi}.png}\\%
}}
\foreach \m in {01,02,03,04,05,06,07,08,09,10,11,12}{\plotter{SH}{\m}}
%\foreach \m in {01,03,05,07,09,11}{\plotter{SH}{\m}}
\renewcommand{\pwidth}{0.3\textwidth}
\renewcommand{\plotter}[2]{% #1=view, #2=date
\frame{
\plotnocbar[\pwidth]{figs/#1_aice_m_mm_1deg_jra55_iaf_omip2_cycle3-obs_1993-2017_mean_month_#2_200dpi.png}%
\plotnocbar[\pwidth]{{figs/#1_aice_m_mm_1deg_jra55_iaf_ensemble_albicev_0.9_cycle1-obs_1993-2017_mean_month_#2_200dpi}.png}%
\plotnocbar[\pwidth]{{figs/#1_aice_m_mm_1deg_jra55_iaf_ensemble_albicev_0.78_cycle1-obs_1993-2017_mean_month_#2_200dpi}.png}%
\cbarrot[\pwidth]{{figs/#1_aice_m_mm_1deg_jra55_iaf_ensemble_albicev_0.9_cycle1-obs_1993-2017_mean_month_#2_200dpi}.png}\\%
\plotnocbar[\pwidth]{{figs/#1_aice_m_mm_025deg_jra55_iaf_omip2_cycle3-obs_1993-2017_mean_month_#2_200dpi}.png}%
\plotnocbar[\pwidth]{{figs/#1_aice_m_mm_025deg_jra55_iaf_ensemble_albicev_0.9_cycle1-obs_1993-2017_mean_month_#2_200dpi}.png}%
\cbarrot[\pwidth]{{figs/#1_aice_m_mm_025deg_jra55_iaf_omip2_cycle3-obs_1993-2017_mean_month_#2_200dpi}.png}\\%
}}
\foreach \m in {01,02,03,04,05,06,07,08,09,10,11,12}{\plotter{SH}{\m}}
%\foreach \m in {01,03,05,07,09,11}{\plotter{SH}{\m}}
%\sicsixpanelspertcompare{SH}{aice_m_mm_1deg_jra55_iaf_ensemble}{_albicev_0.78}{_albicev_0.9}{}{_albicev_0.78_-control}{_albicev_0.9_-control}{01}
%\sicsixpanelspertcompare{SH}{aice_m_mm_1deg_jra55_iaf_ensemble}{_albicev_0.78}{_albicev_0.9}{}{_albicev_0.78_-control}{_albicev_0.9_-control}{03}
%\sicsixpanelspertcompare{SH}{aice_m_mm_1deg_jra55_iaf_ensemble}{_albicev_0.78}{_albicev_0.9}{}{_albicev_0.78_-control}{_albicev_0.9_-control}{05}
%\sicsixpanelspertcompare{SH}{aice_m_mm_1deg_jra55_iaf_ensemble}{_albicev_0.78}{_albicev_0.9}{}{_albicev_0.78_-control}{_albicev_0.9_-control}{07}
%\sicsixpanelspertcompare{SH}{aice_m_mm_1deg_jra55_iaf_ensemble}{_albicev_0.78}{_albicev_0.9}{}{_albicev_0.78_-control}{_albicev_0.9_-control}{09}
%\sicsixpanelspertcompare{SH}{aice_m_mm_1deg_jra55_iaf_ensemble}{_albicev_0.78}{_albicev_0.9}{}{_albicev_0.78_-control}{_albicev_0.9_-control}{11}
\frame{
\frametitle{albicev summary}
\begin{itemize}
\item Bias is slightly weaker at 0.25$^\circ$ but has the same pattern as at 1$^\circ$
\item Perturbation pattern changes only in amplitude as albicev decreases
\item Perturbation pattern has uniform sign in summer and early autumn, with SIC decreasing (increasing) for albicev smaller (larger) than control value (0.86)
\item Perturbation pattern is more complex in other seasons, but the effect of albicev is small in the polar night; cancellation reduces signal in total area timeseries. 
\end{itemize}
}

\subsection{albicei}

\frame{
\frametitle{albicei: ice infrared albedo}
\includegraphics[width=\textwidth]{figs/ice_area_min_mean_max_albicei.pdf}
}

\frame{
\frametitle{Sea ice mass tendency terms}
\includegraphics[width=.49\textwidth]{{figs/SH_1deg_jra55_iaf_ensemble_albicei_0.47_cycle1_1993-2017_mass_terms_seasonal_clim}.pdf}
\includegraphics[width=.49\textwidth]{{figs/SH_1deg_jra55_iaf_ensemble_albicei_0.36_cycle1_1993-2017_mass_terms_seasonal_clim}.pdf}\\
dashed is control
}

\renewcommand{\pwidth}{0.3\textwidth}
\renewcommand{\plotter}[2]{% #1=view, #2=date
\frame{
\plotnocbar[\pwidth]{figs/#1_aice_m_mm_1deg_jra55_iaf_omip2_cycle3-obs_1993-2017_mean_month_#2_200dpi.png}%
\plotnocbar[\pwidth]{{figs/#1_aice_m_mm_1deg_jra55_iaf_ensemble_albicei_0.47_cycle1-obs_1993-2017_mean_month_#2_200dpi}.png}%
\plotnocbar[\pwidth]{{figs/#1_aice_m_mm_1deg_jra55_iaf_ensemble_albicei_0.36_cycle1-obs_1993-2017_mean_month_#2_200dpi}.png}%
\cbarrot[\pwidth]{{figs/#1_aice_m_mm_1deg_jra55_iaf_ensemble_albicei_0.47_cycle1-obs_1993-2017_mean_month_#2_200dpi}.png}\\%
\hspace{\pwidth}%
\plotnocbar[\pwidth]{{figs/#1_aice_m_mm_1deg_jra55_iaf_ensemble_albicei_0.47_-control_1993-2017_mean_month_#2_200dpi}.png}%
\plotnocbar[\pwidth]{{figs/#1_aice_m_mm_1deg_jra55_iaf_ensemble_albicei_0.36_-control_1993-2017_mean_month_#2_200dpi}.png}%
\cbarrot[\pwidth]{{figs/#1_aice_m_mm_1deg_jra55_iaf_ensemble_albicei_0.47_-control_1993-2017_mean_month_#2_200dpi}.png}\\%
}}
\foreach \m in {01,02,03,04,05,06,07,08,09,10,11,12}{\plotter{SH}{\m}}
%\foreach \m in {01,03,05,07,09,11}{\plotter{SH}{\m}}
%\sicsixpanelspertcompare{SH}{aice_m_mm_1deg_jra55_iaf_ensemble}{_albicei_0.36}{_albicei_0.47}{}{_albicei_0.36_-control}{_albicei_0.47_-control}{01}
%\sicsixpanelspertcompare{SH}{aice_m_mm_1deg_jra55_iaf_ensemble}{_albicei_0.36}{_albicei_0.47}{}{_albicei_0.36_-control}{_albicei_0.47_-control}{03}
%\sicsixpanelspertcompare{SH}{aice_m_mm_1deg_jra55_iaf_ensemble}{_albicei_0.36}{_albicei_0.47}{}{_albicei_0.36_-control}{_albicei_0.47_-control}{05}
%\sicsixpanelspertcompare{SH}{aice_m_mm_1deg_jra55_iaf_ensemble}{_albicei_0.36}{_albicei_0.47}{}{_albicei_0.36_-control}{_albicei_0.47_-control}{07}
%\sicsixpanelspertcompare{SH}{aice_m_mm_1deg_jra55_iaf_ensemble}{_albicei_0.36}{_albicei_0.47}{}{_albicei_0.36_-control}{_albicei_0.47_-control}{09}
%\sicsixpanelspertcompare{SH}{aice_m_mm_1deg_jra55_iaf_ensemble}{_albicei_0.36}{_albicei_0.47}{}{_albicei_0.36_-control}{_albicei_0.47_-control}{11}
\frame{
\frametitle{albicei summary}
\begin{itemize}
\item Perturbation pattern changes only in amplitude as albicei decreases
\item Perturbation pattern has uniform sign in summer and early autumn, with SIC decreasing (increasing) for albicei smaller (larger) than control value (0.44)
\item Perturbation pattern is more complex in other seasons, but the effect of albicei is small in the polar night; cancellation reduces signal in total area timeseries. 
\end{itemize}
}

\section{Summary}

\frame{
\frametitle{Perturbation summary}
\begin{itemize}
\item Sensitivity to the parameters tested is generally weak over the range of values explored
\item Sensitivity to these parameters is generally stronger in summer
\item Perturbation patterns generally don't resemble patterns of the seasonal climatological bias relative to SIC observations, and so cannot remove much of the bias
\item No parameter variations tested made much difference to the overall bias
\end{itemize}
}


\frame{
\frametitle{Conclusions}
\begin{itemize}
\item New ACCESS-OM2 configurations, run for 366\,yr at 1$^\circ$ \& 0.25$^\circ$, 183\,yr at 0.1$^\circ$
\item Sea ice extent slightly improved, and much less biased than many OMIP-2 models
\item Resolution dependence is smaller than bias --- so model improvements at low resolution can benefit all resolutions
\item Sea ice distribution is very similar in different cycles, indicating dominance by atmospheric forcing with little role for ocean variability
\item No silver bullets --- model bias not significantly improved by any parameter changes investigated so far
\item Bias may be due to JRA55-do forcing (but biases in JRA55-do do not seem to be due to the COBE-SST SIC assimilated by JRA55)
\item Is this the best we can do with this forcing product? Should we try other parameters? Or other forcing products?
\end{itemize}
}

\end{document} 


\newcommand{\siccomparefourall}[4]{% #1=view, #2=experiment1, #3=experiment2, #4=experiment3
\renewcommand{\years}{\monthlymean}
\siccomparefour{#1}{#2}{#3}{#4}{01}{1993--2017\\[-1ex]January mean}{SI concentration}
\siccomparefour{#1}{#2}{#3}{#4}{02}{1993--2017\\[-1ex]February mean}{SI concentration}
\siccomparefour{#1}{#2}{#3}{#4}{03}{1993--2017\\[-1ex]March mean}{SI concentration}
\siccomparefour{#1}{#2}{#3}{#4}{04}{1993--2017\\[-1ex]April mean}{SI concentration}
\siccomparefour{#1}{#2}{#3}{#4}{05}{1993--2017\\[-1ex]May mean}{SI concentration}
\siccomparefour{#1}{#2}{#3}{#4}{06}{1993--2017\\[-1ex]June mean}{SI concentration}
\siccomparefour{#1}{#2}{#3}{#4}{07}{1993--2017\\[-1ex]July mean}{SI concentration}
\siccomparefour{#1}{#2}{#3}{#4}{08}{1993--2017\\[-1ex]August mean}{SI concentration}
\siccomparefour{#1}{#2}{#3}{#4}{09}{1993--2017\\[-1ex]September mean}{SI concentration}
\siccomparefour{#1}{#2}{#3}{#4}{10}{1993--2017\\[-1ex]October mean}{SI concentration}
\siccomparefour{#1}{#2}{#3}{#4}{11}{1993--2017\\[-1ex]November mean}{SI concentration}
\siccomparefour{#1}{#2}{#3}{#4}{12}{1993--2017\\[-1ex]December mean}{SI concentration}
}

%\siccomparefour{SH}{1deg_jra55_iaf_omip2-fixed_cycle1}{025deg_jra55_iaf_amoctopo_cycle1}{01deg_jra55v140_iaf_cycle1}{01}{1993--2017\\[-1ex]January mean}{SI concentration}

\siccomparefourall{SH}{1deg_jra55_iaf_omip2-fixed_cycle1}{025deg_jra55_iaf_amoctopo_cycle1}{01deg_jra55v140_iaf_cycle1}

%\end{document}


%\end{document}

\frame{
\frametitle{Extra slides}
}

\frame{
\frametitle{Resolution dependence of thickness monthly climatology}
%\begin{itemize}
%\item Plots: 25-year 1993--2017 monthly thickness climatology for 1st cycle at 3 resolutions, compared to the \href{http://psc.apl.washington.edu/zhang/Global_seaice/model.html}{GIOMAS} model output (which assimilates satellite sea ice concentration and is forced by daily NCEP-NCAR reanalysis)
%%\item \textbf{can someone recommend a gridded Antarctic sea ice thickness observational estimate I could use for comparison?}  e.g.\ GIOMAS? ICESat?
%\item Resolution dependence is small, especially between 0.25$^\circ$ and 0.1$^\circ$
%\item 1$^\circ$ extent and thickness are lower than at 0.25$^\circ$ and 0.1$^\circ$
%\item Resolution dependence is stronger in summer than winter; in winter it seems largest in East Antarctic
%\item Area covered by $>1.5$m thickness is much smaller in ACCESS-OM2 at all resolutions than in GIOMAS, especially in summer
%\end{itemize}
}

\sicfourpanelsall{SH}{hi_m_mm}{GIOMAS}{1deg_jra55_iaf_omip2-fixed_cycle1}{025deg_jra55_iaf_amoctopo_cycle1}{01deg_jra55v140_iaf_cycle1}

%\sicthreepanelsall{SH}{hi_m_mm}{1deg_jra55_iaf_omip2-fixed_cycle1}{025deg_jra55_iaf_amoctopo_cycle1}{01deg_jra55v140_iaf_cycle1}


\frame{
\frametitle{Timeseries of sea ice area, cycle \only<1>{1}\only<2>{2}\only<3>{3}\only<4>{4}\only<5>{5}\only<6>{6}}
\includegraphics<1>[height=0.5\textheight,width=\textwidth]{figs/ice_area_min_mean_max_all_cycle1.pdf}%
\includegraphics<2>[height=0.5\textheight,width=\textwidth]{figs/ice_area_min_mean_max_all_cycle2.pdf}%
\includegraphics<3>[height=0.5\textheight,width=\textwidth]{figs/ice_area_min_mean_max_all_cycle3.pdf}%
\includegraphics<4>[height=0.5\textheight,width=\textwidth]{figs/ice_area_min_mean_max_all_cycle4.pdf}%
\includegraphics<5>[height=0.5\textheight,width=\textwidth]{figs/ice_area_min_mean_max_all_cycle5.pdf}%
\includegraphics<6>[height=0.5\textheight,width=\textwidth]{figs/ice_area_min_mean_max_all_cycle6.pdf}%
\begin{itemize}
\item 12-month running minimum, mean and maximum area, compared to observational estimate (\href{https://nsidc.org/data/g02135}{NSIDC Sea Ice Index, v3; Fetterer et al.})
\item Very close tracking of interannual variation in obs, due to data assimilation into JRA55-do reanalysis: \textbf{2016 Antarctic sea ice minimum is captured}
\item Little dependence on cycle number, apart from initial few decades at 1$^\circ$ and 0.25$^\circ$
\item Quantitative comparison is hampered by differing land masks
\end{itemize}
}

\frame{
\frametitle{Timeseries of sea ice volume, cycle \only<1>{1}\only<2>{2}\only<3>{3}\only<4>{4}\only<5>{5}\only<6>{6}}
\includegraphics<1>[height=0.5\textheight,width=\textwidth]{figs/ice_volume_min_mean_max_all_cycle1.pdf}%
\includegraphics<2>[height=0.5\textheight,width=\textwidth]{figs/ice_volume_min_mean_max_all_cycle2.pdf}%
\includegraphics<3>[height=0.5\textheight,width=\textwidth]{figs/ice_volume_min_mean_max_all_cycle3.pdf}%
\includegraphics<4>[height=0.5\textheight,width=\textwidth]{figs/ice_volume_min_mean_max_all_cycle4.pdf}%
\includegraphics<5>[height=0.5\textheight,width=\textwidth]{figs/ice_volume_min_mean_max_all_cycle5.pdf}%
\includegraphics<6>[height=0.5\textheight,width=\textwidth]{figs/ice_volume_min_mean_max_all_cycle6.pdf}%
\begin{itemize}
\item 12-month running minimum, mean and maximum volume
\item Little dependence on cycle number, apart from initial few decades at 1$^\circ$ and 0.25$^\circ$
\end{itemize}
}


\frame{
\frametitle{1993--2017 mean annual cycle of sea ice extent}
\includegraphics[width=\textwidth]{figs/ice_extent_seasonal_clim_cycle1.pdf}\\
{\scriptsize Obs: NOAA/NSIDC G02135 Sea Ice Index v3 (Fetterer et al., 2017)}
}

\frame{
\frametitle{1993--2017 mean annual cycle of sea ice area}
\includegraphics[width=\textwidth]{figs/ice_area_seasonal_clim_cycle1.pdf}\\
{\scriptsize Obs: NOAA/NSIDC G02135 Sea Ice Index v3 (Fetterer et al., 2017)}
}


\frame{
\frametitle{Data sources}
\begin{itemize}
%\item Runs used new configurations that are more consistent than in \href{https://doi.org/10.5194/gmd-13-401-2020}{Kiss et al.\ (2020)}
%\item Each cycle had 61 years of JRA55-do v1.4 forcing, 1 Jan 1958 -- 31 Dec 2018
%\item Cycle 1 initial cond.\ is WOA13 v2; for cycle $n>1$, IC is final state of cycle $n-1$ 
\item ACCESS-OM2: Hakase's 6-cycle OMIP-2 run at 1$^\circ$ \texttt{/scratch/v45/aek156/access-om2/archive/1deg_jra55_iaf_omip2-fixed_cycle*}
\item ACCESS-OM2-025: Ryan's 6-cycle OMIP-2 run at  0.25$^\circ$ \texttt{/scratch/e14/rmh561/access-om2/archive/025deg_jra55_iaf_amoctopo_cycle*}
\item ACCESS-OM2-01: 3 cycles at 0.1$^\circ$ \texttt{/g/data/cj50/access-om2/raw-output/access-om2-01/01deg_jra55v140_iaf}, \texttt{/g/data/cj50/access-om2/raw-output/access-om2-01/01deg_jra55v140_iaf_cycle2}, \texttt{/scratch/x77/aek156/access-om2/archive/01deg_jra55v140_iaf_cycle3} and \texttt{/scratch/v45/aek156/access-om2/archive/01deg_jra55v140_iaf_cycle3}
\item Perturbation experiments: \texttt{/home/156/aek156/payu/param_ensemble}
\item Perturbation ensemble: \url{https://github.com/aekiss/ensemble/blob/1deg_param_ensemble/ensemble.yaml}
\end{itemize}
}

\frame{
\frametitle{Comparisons to JRA55-do forcing fields}
}

\sicsixpanelsJRAcompare{Amundsen-BellingshausenSP}{Amundsen-Bellingshausen_aice_01deg_jra55v140_iaf_cycle1}{tas}{rlds}{rsds}{psl}{uas}{2017-01-01}
\sicsixpanelsJRAcompare{Amundsen-BellingshausenSP}{Amundsen-Bellingshausen_aice_01deg_jra55v140_iaf_cycle1}{tas}{rlds}{rsds}{psl}{uas}{2017-01-02}
\sicsixpanelsJRAcompare{Amundsen-BellingshausenSP}{Amundsen-Bellingshausen_aice_01deg_jra55v140_iaf_cycle1}{tas}{rlds}{rsds}{psl}{uas}{2017-01-03}
\sicsixpanelsJRAcompare{Amundsen-BellingshausenSP}{Amundsen-Bellingshausen_aice_01deg_jra55v140_iaf_cycle1}{tas}{rlds}{rsds}{psl}{uas}{2017-01-04}
\sicsixpanelsJRAcompare{Amundsen-BellingshausenSP}{Amundsen-Bellingshausen_aice_01deg_jra55v140_iaf_cycle1}{tas}{rlds}{rsds}{psl}{uas}{2017-01-05}
\sicsixpanelsJRAcompare{Amundsen-BellingshausenSP}{Amundsen-Bellingshausen_aice_01deg_jra55v140_iaf_cycle1}{tas}{rlds}{rsds}{psl}{uas}{2017-01-06}
\sicsixpanelsJRAcompare{Amundsen-BellingshausenSP}{Amundsen-Bellingshausen_aice_01deg_jra55v140_iaf_cycle1}{tas}{rlds}{rsds}{psl}{uas}{2017-01-07}
\sicsixpanelsJRAcompare{Amundsen-BellingshausenSP}{Amundsen-Bellingshausen_aice_01deg_jra55v140_iaf_cycle1}{tas}{rlds}{rsds}{psl}{uas}{2017-01-08}
\sicsixpanelsJRAcompare{Amundsen-BellingshausenSP}{Amundsen-Bellingshausen_aice_01deg_jra55v140_iaf_cycle1}{tas}{rlds}{rsds}{psl}{uas}{2017-01-09}
\sicsixpanelsJRAcompare{Amundsen-BellingshausenSP}{Amundsen-Bellingshausen_aice_01deg_jra55v140_iaf_cycle1}{tas}{rlds}{rsds}{psl}{uas}{2017-01-10}
\sicsixpanelsJRAcompare{Amundsen-BellingshausenSP}{Amundsen-Bellingshausen_aice_01deg_jra55v140_iaf_cycle1}{tas}{rlds}{rsds}{psl}{uas}{2017-01-11}
\sicsixpanelsJRAcompare{Amundsen-BellingshausenSP}{Amundsen-Bellingshausen_aice_01deg_jra55v140_iaf_cycle1}{tas}{rlds}{rsds}{psl}{uas}{2017-01-12}
\sicsixpanelsJRAcompare{Amundsen-BellingshausenSP}{Amundsen-Bellingshausen_aice_01deg_jra55v140_iaf_cycle1}{tas}{rlds}{rsds}{psl}{uas}{2017-01-13}
\sicsixpanelsJRAcompare{Amundsen-BellingshausenSP}{Amundsen-Bellingshausen_aice_01deg_jra55v140_iaf_cycle1}{tas}{rlds}{rsds}{psl}{uas}{2017-01-14}
\sicsixpanelsJRAcompare{Amundsen-BellingshausenSP}{Amundsen-Bellingshausen_aice_01deg_jra55v140_iaf_cycle1}{tas}{rlds}{rsds}{psl}{uas}{2017-01-15}
\sicsixpanelsJRAcompare{Amundsen-BellingshausenSP}{Amundsen-Bellingshausen_aice_01deg_jra55v140_iaf_cycle1}{tas}{rlds}{rsds}{psl}{uas}{2017-01-16}
\sicsixpanelsJRAcompare{Amundsen-BellingshausenSP}{Amundsen-Bellingshausen_aice_01deg_jra55v140_iaf_cycle1}{tas}{rlds}{rsds}{psl}{uas}{2017-01-17}
\sicsixpanelsJRAcompare{Amundsen-BellingshausenSP}{Amundsen-Bellingshausen_aice_01deg_jra55v140_iaf_cycle1}{tas}{rlds}{rsds}{psl}{uas}{2017-01-18}
\sicsixpanelsJRAcompare{Amundsen-BellingshausenSP}{Amundsen-Bellingshausen_aice_01deg_jra55v140_iaf_cycle1}{tas}{rlds}{rsds}{psl}{uas}{2017-01-19}
\sicsixpanelsJRAcompare{Amundsen-BellingshausenSP}{Amundsen-Bellingshausen_aice_01deg_jra55v140_iaf_cycle1}{tas}{rlds}{rsds}{psl}{uas}{2017-01-20}


\frame{
\frametitle{SIC monthly climatology in different cycles}
\begin{itemize}
\item Plots: 25-year 1993--2017 monthly climatology for each cycle, and difference between cycle $n$ and cycle 1 (omitting cycle 6)
\item Consider the cycles as an ensemble: identical configuration and forcing, differing only in initial condition and therefore in ocean state
\item Differences between cycles are very small: atmospheric forcing is dominant and differing ocean conditions have a secondary role
\item Difference between cycle 1 and 2 is usually the largest
\item Difference pattern is very similar between cycles and resolutions, and often grows slightly through the first few cycles, in many cases slightly worsening the bias relative to obs --- probably reflects model drift
\end{itemize}
}

\siccomparefour{SH}{1deg_jra55_iaf_omip2-fixed_cycle1}{025deg_jra55_iaf_amoctopo_cycle1}{01deg_jra55v140_iaf_cycle1}{03}{1993--2017\\[-1ex]March mean}{SI concentration}

\siccomparefour{SH}{1deg_jra55_iaf_omip2-fixed_cycle2}{025deg_jra55_iaf_amoctopo_cycle2}{01deg_jra55v140_iaf_cycle2}{03}{1993--2017\\[-1ex]March mean}{SI concentration}

\siccomparefour{SH}{1deg_jra55_iaf_omip2-fixed_cycle3}{025deg_jra55_iaf_amoctopo_cycle3}{01deg_jra55v140_iaf_cycle3}{03}{1993--2017\\[-1ex]March mean}{SI concentration}

\siccomparefour{SH}{1deg_jra55_iaf_omip2-fixed_cycle4}{025deg_jra55_iaf_amoctopo_cycle4}{01deg_jra55v140_iaf_cycle3}{03}{1993--2017\\[-1ex]March mean}{SI concentration}

\siccomparefour{SH}{1deg_jra55_iaf_omip2-fixed_cycle5}{025deg_jra55_iaf_amoctopo_cycle5}{01deg_jra55v140_iaf_cycle3}{03}{1993--2017\\[-1ex]March mean}{SI concentration}

\siccomparefour{SH}{1deg_jra55_iaf_omip2-fixed_cycle6}{025deg_jra55_iaf_amoctopo_cycle6}{01deg_jra55v140_iaf_cycle3}{03}{1993--2017\\[-1ex]March mean}{SI concentration}


\siccomparefour{SH}{1deg_jra55_iaf_omip2-fixed_cycle1}{025deg_jra55_iaf_amoctopo_cycle1}{01deg_jra55v140_iaf_cycle1}{09}{1993--2017\\[-1ex]September mean}{SI concentration}

\siccomparefour{SH}{1deg_jra55_iaf_omip2-fixed_cycle2}{025deg_jra55_iaf_amoctopo_cycle2}{01deg_jra55v140_iaf_cycle2}{09}{1993--2017\\[-1ex]September mean}{SI concentration}

\siccomparefour{SH}{1deg_jra55_iaf_omip2-fixed_cycle3}{025deg_jra55_iaf_amoctopo_cycle3}{01deg_jra55v140_iaf_cycle3}{09}{1993--2017\\[-1ex]September mean}{SI concentration}

\siccomparefour{SH}{1deg_jra55_iaf_omip2-fixed_cycle4}{025deg_jra55_iaf_amoctopo_cycle4}{01deg_jra55v140_iaf_cycle3}{09}{1993--2017\\[-1ex]September mean}{SI concentration}

\siccomparefour{SH}{1deg_jra55_iaf_omip2-fixed_cycle5}{025deg_jra55_iaf_amoctopo_cycle5}{01deg_jra55v140_iaf_cycle3}{09}{1993--2017\\[-1ex]September mean}{SI concentration}

\siccomparefour{SH}{1deg_jra55_iaf_omip2-fixed_cycle6}{025deg_jra55_iaf_amoctopo_cycle6}{01deg_jra55v140_iaf_cycle3}{09}{1993--2017\\[-1ex]September mean}{SI concentration}


\siccomparefivecyclesdiff{SH}{1deg_jra55_iaf_omip2-fixed}{03}

\siccomparefivecyclesdiff{SH}{025deg_jra55_iaf_amoctopo}{03}

\siccomparethreecyclesdiff{SH}{01deg_jra55v140_iaf}{03}


\siccomparefivecyclesdiff{SH}{1deg_jra55_iaf_omip2-fixed}{09}

\siccomparefivecyclesdiff{SH}{025deg_jra55_iaf_amoctopo}{09}

\siccomparethreecyclesdiff{SH}{01deg_jra55v140_iaf}{09}


\frame{
\frametitle{Arctic}
}
\siccomparefourall{NH}{1deg_jra55_iaf_omip2-fixed_cycle1}{025deg_jra55_iaf_amoctopo_cycle1}{01deg_jra55v140_iaf_cycle1}

%\siccomparefour{SH}{1deg_jra55_iaf_omip2-fixed_cycle1}{025deg_jra55_iaf_amoctopo_cycle1}{01deg_jra55v140_iaf_cycle1}{11}
\frame{
\frametitle{Resolution dependence of SIC monthly climatology}
\begin{itemize}
\item Plots: 25-year 1993--2017 monthly sea ice concentration climatology for 1st cycle at 3 resolutions and obs
\item Obs: NOAA G02202 V3 passive microwave Goddard merged monthly data
\item Scale is quadratic to emphasise differences at high concentration
\item Model SIC is much too low at summer minimum (Feb), especially at 1$^\circ$; ice absent in East~Antarctic
\item Autumn growth looks good, especially at 0.1$^\circ$
\item Sept maximum SIC too low at 1$^\circ$; better at higher res but still a bit low in outer pack and too high near coast. Apparent coastal polynyas in obs are not seen at 0.1$^\circ$ (in fact the opposite, with high coastal SIC). Might be an observational bias, as mentioned in GMD paper (p433)?
\item Spring decay too rapid
\item Models agree better with each other than with obs: bias is mostly resolution-independent
\end{itemize}
}


\frame{
\frametitle{Example: daily sea ice concentration in Amundsen-Bellingshausen Sea}
Comparing the three resolutions
\begin{itemize}
\item Close agreement between different model resolutions, despite differing ocean states
\item Ice is stirred by partially-resolved ocean eddy field at 0.1$^\circ$
\item \ldots but models have only a weak resemblance to the microwave retrievals
\item Common biases, independent of ocean state $\Rightarrow$ JRA55-do bias or poor model parameter choices?
\end{itemize}
}


\frame{
\frametitle{Example: daily sea ice concentration in Amundsen-Bellingshausen Sea}
Comparing 3 cycles of the 0.1$^\circ$ experiment
\begin{itemize}
\item Close agreement between different model cycles, despite differing ocean states
\item Ice is stirred by partially-resolved ocean eddy field at 0.1$^\circ$
\item \ldots but models have only a weak resemblance to the microwave retrievals
\item Common biases, independent of ocean state $\Rightarrow$ JRA55-do bias or poor model parameter choices?
\end{itemize}
}


\frame{
\frametitle{Resolution dependence of thickness monthly climatology}
\begin{itemize}
\item Plots: 25-year 1993--2017 monthly thickness climatology for 1st cycle at 3 resolutions, compared to the \href{http://psc.apl.washington.edu/zhang/Global_seaice/model.html}{GIOMAS} model output (which assimilates satellite sea ice concentration and is forced by daily NCEP-NCAR reanalysis)
%\item \textbf{can someone recommend a gridded Antarctic sea ice thickness observational estimate I could use for comparison?}  e.g.\ GIOMAS? ICESat?
\item Resolution dependence is small, especially between 0.25$^\circ$ and 0.1$^\circ$
\item 1$^\circ$ extent and thickness are lower than at 0.25$^\circ$ and 0.1$^\circ$
\item Resolution dependence is stronger in summer than winter; in winter it seems largest in East Antarctic
\item Area covered by $>1.5$m thickness is much smaller in ACCESS-OM2 at all resolutions than in GIOMAS, especially in summer
\end{itemize}
}

\frame{
\frametitle{To do?}
\begin{itemize}
%\item include 15\% SIC contours on plots
\item superimpose JRA data (wind vectors, isobars, isotherms)
\item compare thickness with obs estimate, e.g.\ ICESat? --- see \url{https://earth.gsfc.nasa.gov/index.php/cryo/data/antarctic-sea-ice-thickness}
%\item plot timeseries of extent and volume
%\item plot mean annual cycle of extent
\item plot SST biases
%\item some nice daily snapshots showing the ice behaviour at different resolutions 
\item analyses using daily data? e.g.\ thickness variance at different resolution?
\item Taylor diagrams from Will?
\item Plots focusing on 2016 sea ice event - timeseries and maps
\item cf.\ \href{https://doi.org/10.5194/gmd-13-3643-2020}{Tsujino et al 2020} figs 4, 9, 22, 23 \& table 2. 
Figs 9 \& 22 show a substantial difference between CORE and JRA55-do,  with JRA reducing (but not eliminating) the magnitude of the March low bias. The impact in Sept is less clear and depends on the model, but Fig 23 shows significantly improved Sept interannual variation with JRA.
\item \textbf{any other suggestions?} e.g.\ diagnostics that would give dynamical insight?
\end{itemize}
}

\end{document}
